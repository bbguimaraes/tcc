\documentclass{iiufrgs}

\usepackage[utf8]{inputenc}
\usepackage[T1]{fontenc}
\usepackage[official]{eurosym}
\usepackage{color}
\usepackage{graphicx}
\usepackage{amsmath}
\usepackage{listings}
\usepackage{hyperref}
\usepackage{multirow}
\usepackage{rotating}
\usepackage{enumerate}
\usepackage[table]{xcolor}
\usepackage{hhline}
\usepackage{array}

\usepackage{setspace}
\onehalfspacing

\def\lstlistlistingname{Lista de Trechos de Código}
\def\lstlistingname{Trecho de Código}
\lstset{
    inputencoding=utf8,
    extendedchars=true,
    basicstyle=\scriptsize\color{black}\ttfamily,
    numbers=left,
    numberstyle=\scriptsize\ttfamily,
    stepnumber=1,
    numbersep=10pt,
    tabsize=4,
    frame=single,
    keywordstyle=\color{black}\textbf,
    keywordstyle=[1]\color{black}\textbf,
    keywordstyle=[2]\color{black}\textbf,
    keywordstyle=[3]\color{black}\textbf,
    keywordstyle=[4]\color{black}\textbf,
    stringstyle=\color{black}\ttfamily,
    showspaces=false,
    showtabs=false,
    xleftmargin=17pt,
    framextopmargin=4pt,
    framexleftmargin=25pt,
    framexrightmargin=5pt,
    framexbottommargin=4pt,
    captionpos=b,
    showstringspaces=false,
    literate={á}{{\'a}}1 {é}{{\'e}}1 {í}{{\'i}}1 {ó}{{\'o}}1 {ç}{{\c{c}}}1 {ã}{{\~a}}1 {ê}{{\^e}}1 {ú}{{\'u}}1,
}

\title{Mineração de Dados baseada nos Sistemas Imunológicos: um estudo de caso na Detecção de Fraude}
\author{Guimarães}{Bruno Barcarol}
\advisor{Webber}{Carine Geltrudes}
\course{\cgcc}
\location{Caxias do Sul}{}

\keyword{mineração de dados}
\keyword{inteligência artificial}
\keyword{sistemas imunológicos artificiais}
\keyword{fraude}
\keyword{detecção de fraude}
\keyword{segurança da informação}

% Documento.
\begin{document}

% Folha de rosto.
\maketitle

% Sumário.
\tableofcontents

% Lista de figuras.
\listoffigures

\begin{abstract}
    Os Sistemas Imunológicos Artificiais são um campo da Inteligência Artificial que se desenvolveu no início dos anos 90, e continua sendo objeto de pesquisa até hoje. Iniciando na área de segurança de computadores, os algoritmos baseados no sistema imunológico foram utilizados em diversas áreas da computação. O poder de abstração da modelagem dos elementos do sistema como componentes do sistema imunológico natural ajudaram no desenvolvimento desse sistemas. Utilizando um pacote de algoritmos imunológicos e a ferramenta WEKA, esse trabalho tem o objetivo de verificar como a utilização de Sistemas Imunológicos Artificiais influencia um sistema computacional, comparando-os com técnicas mais tradicionais da Mineração de Dados e da Inteligência Artificial.
\end{abstract}

\begin{englishabstract}{Immune System based Data Mining: a case study on Fraud Detection}{mineração de dados, inteligência artificial, sistemas imunológicos artificiais, fraude, detecção de fraude, segurança da informação}
    Artificial Immune Systems are a field of Artificial Intelligence that developed on the early 90's, and remains subject to researches event today. Beginning on computer security systems, algorithms based on the immune system have been used on many areas of computing. The power of abstraction the the design of these systems as components of the natural immune system helped their development. Using a package of immune algorithms and the WEKA environment, this study will verify how the application of Artificial Immune Systems influences a computational system, comparing them with more traditional techniques from Data Mining and Artificial Intelligence.
\end{englishabstract}

% Conteúdo.
\chapter{Introdução}

O crescente avanço na tecnologia de armazenamento de dados, principalmente em termos de velocidade e tamanho, permite a criação de bancos de dados complexos e detalhados. Com isso, o desenvolvimento de programas de computador que há anos atrás seriam computacionalmente inviáveis torna-se possível. Uma área fortemente influenciada por esse fator é a mineração de dados.

Definida como ``análise de bases (geralmente extensas) de dados previamente coletados para estabelecer relações e apresentar a informação de forma mais clara e útil ao proprietário dos dados'' \cite[p. 1]{Hand2001}\footnote{``Data mining is the analysis of (often large) observational data sets to find unsuspected relationships and to summarize the data in novel ways that are both understandable and useful to the data owner.''.}, a mineração de dados é associada à ideia do aproveitamento de grandes bases de dados para auxiliar o profissional em alguma tarefa, através da apresentação de padrões e relações que seriam difíceis ou impossíveis de serem encontrados pelo observador humano. As atividades envolvidas na mineração de dados são:

\begin{itemize}
\item Determinar como os dados a serem usados serão representados. As estruturas do problema devem ser codificadas para que possam ser utilizadas por algoritmos computacionais. Formas comuns de representação são cadeias de bits, números inteiros, números reais e valores categóricos.
\item Decidir a função de aptidão. Essa função é utilizada para quantificar a adequação de um modelo a um conjunto de dados, permitindo definir se um modelo é "melhor" que outro, ou seja, representa mais precisamente os elementos daquele conjunto.
\item Escolher um processo algorítmico para otimizar a função de comparação. Através de um algoritmo específico, modelos diferentes (ou parâmetros diferentes para um mesmo tipo de modelo) são criados e testados utilizando a função de aptidão. Esse processo é repetido até que a condição de término seja encontrada, geralmente após um determinado limite da função de aptidão ou um determinado número de iterações.
\item Administrar o acesso aos dados de forma eficiente durante a execução dos algoritmos. Grandes \emph{datasets} geralmente excedem o tamanho dos dispositivos de armazenamento primário atuais. Dispositivos de armazenamento secundário ou terciário são utlizados, tornando o acesso a esses dados ineficiente. Um algoritmo corre o risco de tornar-se computacionalmente inviável caso o programador não considere esse fato.
\end{itemize}

Os exemplos da sua utilização são inúmeros, estendendo-se virtualmente a qualquer atividade, de empresas comerciais à medicina e engenharia. A coleta e armazenamento de informações cresce constantemente, tornando a análise desses dados impossível sem o uso de uma ferramenta automatizada. A mineração extrai padrões ou modelos dessas fontes de informação, que seriam difíceis ou impossíveis de serem visualizadas apenas observando-se os dados. Isso tem motivado o desenvolvimento de algoritmos capazes de identificar padrões com mais detalhamento e significado.

Como área interdisciplinar, a mineração atrai estudiosos de diversas áreas da computação como, em especial, a Inteligência Artificial. A combinação de técnicas tradicionais de mineração com os algoritmos da Inteligência Artificial estende em muito o seu poder de análise. Tomando a definição básica de Luger \cite[p. 1]{Luger2009}:

\begin{quote}
``Inteligência Artificial (IA) pode ser definida como a área da ciência da computação que se dedica a automação de comportamento inteligente.''\footnote{``Artificial intelligence (AI) may be defined as the branch of computer science that is concerned with the automation of inteligent behaviour.''.}
\end{quote}

Analisar informações, raciocinar e tirar conclusões são também objeto de estudo da Inteligência Artificial. Redes neurais e bayesianas, clustering e lógica nebulosa (\emph{fuzzy}) são alguns dos métodos mais aplicados nessas situações. De fato, as grandes bases de dados que geralmente são objeto de atuação da mineração constituem excelente campo de testes para os algoritmos da IA. 

Uma área que têm recebido crescente atenção de estudos de mineração de dados e Inteligência Artificial é a de segurança da informação. O desenvolvimento de ferramentas de mineração é visto como um forte candidato à solução desses problemas, já que elas são capazes de extrair modelos e comportamentos da base de dados que dificilmente seriam percebidos por um observador humano. Esses modelos podem ser usados para identificar pardões em novas instâncias, automatizando o processo de monitoramento. O ganho que um profissional tem ao utilizar uma ferramenta dessa natureza é enorme: a tarefa de análise de bases de dados extensas é passada dele para o computador. O reconhecimento de padrões e capacidade de investigação de grandes bases de dados são os maiores desafios que profissionais da segurança da informação enfrentam. Nesse sentido, o objetivo é relevar à maquina cada vez mais o trabalho mecânico, para que o profissional possa se dedicar à análise crítica do material compilado.

Dentro da segurança da informação, escolheu-se um contexto mais específico para delimitar o escopo do trabalho: a detecção de fraude. Na legislação brasileira \cite[p. 324]{DePlacido1982}, a fraude é definida como ``o \emph{engano malicioso} ou a \emph{ação astuciosa}, promovidos de \emph{má fé}, para \emph{ocultação da verdade} ou \emph{fuga ao cumprimento do dever}''. Os tipos mais proeminentes de fraude ocorrem na área médica, nos seguros (de vida, casa, automóveis, etc), cartões de crédito e telecomunicações. Segundo a agência americana de investigação FBI \cite{FBI2010}, o custo das fraudes na área de seguros, excluindo-se os seguros de saúde, apenas nos Estados Unidos, é estimado em 40 bilhões de dólares por ano. O aumento médio nas apólices de uma família comum devido à fraude é calculado em 400 a 700 dólares.


Com frequência, nas ciências, pesquisadores se voltam a áreas diferentes das suas próprias, com o objetivo de encontrar soluções para os seus problemas adaptando outras soluções. Um exemplo disso, na Inteligência Artificial, são os Sistemas Imunológicos Artificiais, inspirados no sistema imunológico natural. Conforme definiu Castro:

\begin{quote}
``Sistemas Imunológicos Artificiais (SIA) são sistemas adaptativos, inspirados na teoria da imunidade e funções, princípios e modelos imunológicos observadas, que são aplicados à resolução de problemas.''\footnote{``Artificial Immune Systems (AIS) are adaptive systems, inspired by theoretical immunology and observed immune functions, principles and models, which are applied to problem solving.''. CASTRO, L. N.; TIMMIS, J. 2002. \emph{Artificial Immune Systems: A New Computational Intelligence Approach.} p. 57.}
\end{quote}

O sistema imunológico natural é um sistema robusto, complexo e adaptativo, que classifica as células do corpo como próprias e não-próprias. Isso é feito através de uma força de trabalho distribuída, capaz de agir de forma local ou global, usando uma rede de mensagens químicas para comunicar-se \cite{Aickelin2005}. Modelos computacionais foram criados para tentar adaptar o funcionamento do sistema imunológico natural a problemas da computação, e dá-se a esses modelos o nome de Sistemas Imunológicos Artificiais. A utilização desse tipo de sistema é relativamente nova: os primeiros trabalhos nessa área foram desenvolvidos a partir da metade dos anos 90. Mesmo assim é uma técnica que apresenta bons resultados e tem despertado o interesse de desenvolvedores de sistemas de segurança da informação. A modelagem desse tipo de sistema como um Sistema Imunológico Artificial oferece diversas vantagens:

\begin{itemize}
\item O sistema imunológico natural protege o corpo de invasões externas. Essa metáfora pode ser facilmente estendida para problemas de segurança, facilitando a modelagem dos componentes do sistema.
\item A natureza distribuída do sistema imunológico facilita a implementação do sistema seguindo o paradigma dos sistemas distribuídos da computação.
\item O sistema imunológico tem características que são muito importantes nos sistemas de segurança da informação. Ele contém uma memória, que reconhece agentes que o corpo já teve contato e apresenta um comportamento aptativo, capaz de reconhecer e combater novas ameaças que ainda não haviam sido tratadas.
\end{itemize}

Uma adição ainda mais recente foi a da Teoria do Perigo, proposta por Matzinger, em 1994. A principal revolução na Teoria do Perigo é a mudança na forma como as células tomam conhecimento e reagem, introduzindo a noção de perigo e tolerância.

\section{Objetivos}
\section{Organização}


\chapter{Sistemas imunológicos e a Teoria do Perigo}

O primeiro modelo do sistema imunológico foi o da distinção entre o próprio e o não-próprio, proposto por Burnet\footnote{BURNET, F. M. 1959. \emph{The Clonal Selection Theory of Acquired Immunity}.}. Ao longo do tempo, novos modelos foram criados, tentando resolver as questões que os outros modelos não explicavam, destancando-se o modelo do não-próprio infeccioso, de Janeway\footnote{JANEWAY, C. A. 1989. \emph{Approaching the Asymptote? Evolution and Revolution in Immunology}.}.

\section{Discriminação do próprio e não-próprio}

Nesse modelo, existia apenas um tipo de linfócito, o linfócito B, repsonsável por identificar os antígenos e produzir os respectivos anticorpos. A resposta imune adaptativa era gerada apenas pela identificação desses antígenos, sem qualquer mecanismo de controle.

\begin{figure}[h]
\centering
\includegraphics[width=0.75\textwidth]{img/signals1-antigen.png}
\caption{Antígeno no controle}
\end{figure}

Em algum ponto no início da vida, os linfócitos aprendiam a diferenciar o próprio, células pertencentes ao corpo, do não-próprio, células estranhas ao corpo, que deveriam ser eliminadas. As células que geravam respostas auto-imunes (reações contra as células próprias) era removidas nesse ponto, restando apenas as capazes de identificar o não-próprio.

\begin{figure}[h]
\centering
\includegraphics[width=0.75\textwidth]{img/signals2-helper.png}
\caption{Auxiliar no controle}
\end{figure}

Bretscher e Cohn introduziram uma segunda célula, o linfócito T auxiliar, ou T$_{h}$ (T \emph{helper}), que regulava a ativação dos linfócitos B. O linfócito B apresenta o antígeno ao linfócito T e aguarda sua confirmação para iniciar a resposta. A introdução dessa célula tem o objetivo de evitar uma reação auto-imune sem controle. Lafferty e Cunningham introduziram uma terceira célula, a Célula Apresentadora de Antígeno (APC, \emph{Antigen Presenting Cell}), cuja função é decompor os antígenos e apresentá-los aos linfócitos T. Dessa forma, o funcionamento das células do modelo anterior agora depende da ativação do linfócito T pela APC.

\begin{figure}[h]
\centering
\includegraphics[width=0.75\textwidth]{img/signals3-apc.png}
\caption{APC no controle}
\end{figure}

\section{Não-próprio infeccioso}

O modelo mais aceito na imunologia atualmente, foi proposto por Janeway. Nele, as APCs também têm de ser ativadas: elas só enviam o sinal dois aos linfócitos T quando tiverem reconhecido padrões patológicos no antígeno. Assim, da mesma forma, o funcionamento do modelo anterior depende de um novo fator, antes a APC, agora o recdonhecimento de padrões pela APC.

\begin{figure}[h]
\centering
\includegraphics[width=0.75\textwidth]{img/signals4-ins.png}
\caption{Não-próprio infeccioso no controle}
\end{figure}

\section{A Teoria do Perigo}

Uma adição ainda mais recente foi a da Teoria do Perigo \cite{Matzinger1994}. Matzinger defende a teoria de que não é a separação do próprio e do não-próprio a força que impulsiona o sistema imunológico, mas sim o que ela caracteriza como ``perigo'': qualquer coisa que cause estresse ou morte não-apoptética (não natural) da célula. Embora não seja completa, essa teoria explica fenômenos que outras teorias sobre o sistema imunológico não explicavam, como o fato de não haver reação contra bactérias no intestino ou na comida, a mudança do conceito de próprio durante a vida e a problemática da própria definição do próprio e do não-próprio.

Mais do que isso, a Teoria do Perigo muda a forma como se enxerga o sistema imunológico como um todo: um sistema responsável por manter o corpo em estado de equilíbrio. Dessa forma, a distinção explícita do próprio se torna desnecessária. Discriminação ainda existe, mas seu foco é o perigo, não mais o estranho.

Uma explicação detalhada do sistema imunológico, e seu funcionamento conforme a Teoria do Perigo, é apresentado na figura 1, abaixo\footnote{MATZINGER, P. \emph{op. cit.}}. Os linfócitos B são os responsáveis pela identificação de antígenos através de receptores em sua superfície. Durante uma infecção, essas células se multiplicam e produzem anticorpos para eliminar os antígenos identificados. Elas são capazes de adaptar-se a virtualmente qualquer tipo de antígeno, gerando a resposta adequada. Um outro tipo de linfócito, os linfócitos T, quando ativados, podem exercer uma de duas funções: os linfócitos T citotóxicos (CTL) são responsáveis pela eliminação de células infectadas, enquanto que os linfócitos T auxiliares (T$_{h}$) são responsáveis pela ativação de outras células, entre elas os linfócitos B.

Além disso, alguns linfócitos T são mantidos como células de memória. Durante a resposta imunológica, os linfócitos B e T$_{h}$ se multiplicam para combater os antígenos, e são removidos quando a resposta termina. No entanto, alguns linfócitos T são mantidos, para que possam ser usados em futuras respostas ao mesmo tipo de antígeno. A combinação de linfóctios T virgens (sem um tipo de antígeno associado) e de memória permite que o sistema imunológico gere respostas tanto a novas ameaças quanto a ameaças recorrentes.

Existe ainda um outro tipo de linfócito, o T \emph{killer} (Tk), que não tem receptores antígeno-específicos, mas é capaz de reconhecer células infectadas e algumas células anormais. O objetivo dessas células é atuar como a primeira defesa contra infecções, e são muito importantes nos começo da vida.

\begin{figure}[h]
\centering
\includegraphics[width=0.75\textwidth]{img/signals5-danger.png}
\caption{Perigo no controle}
\end{figure}

O comportamento dos linfócitos T e B se baseia em três regras:

\begin{enumerate}
\item Linfócitos T e B entram em atividade ao receber os sinais um e dois, morrem ao receber apenas o sinal um e ignoram o sinal dois sem o recebimento do sinal um.
\item Linfócitos T aceitam o sinal dois apenas de APCs, enquanto os linfócitos B, apenas de linfócitos T ativos ou células de memória. O sinal um pode ter origem em qualquer célula.
\item Linfócitos T ativados não precisam do sinal dois para entrarem em ação. Após um período de tempo, elas voltam ao estado de repouso, necessitando novamente dos dois sinais.
\end{enumerate}

Células Apresentadoras de Anítgeno (ATP, ou Antigen Presenting Cell) são as células responsáveis por apresentar os antígenos às células T. Essas células podem ser os linfócitos B, macrófagos e as células dendríticas. Quando as células se encontram em estado de estresse ou morrem de forma não programada, enviam um sinal para as células APC, representado pela seta Sinal 0. Isso desencadeia o envio do sinal dois para os linfócitos Th. Enquanto isso, linfócitos B usam seus receptores para reconhecer antígenos nas suas redondezas, enviando o sinal um para os linfócitos Th. É o par de sinais um (reconhecimento do antígeno) e dois (sinal enviado pelo linfócito T, ativado pelo sinal de perigo) que faz com que a reação imunológica tenha início.

Um conceito importante da Teoria do Perigo é a tolerância. Linfócitos B que reconhecem células próprias continuamente recebem o sinal um (identificação através dos receptores em sua superfície). No entanto, essas células não apresentam perigo (não causam danos a outras células), logo o sinal dois nunca será gerado. Pela lei um acima, sem receber o sinal dois de outras células, o linfócito será removido. Dessa maneira, corpos estranhos mas que não causem dano (perigo) ao corpo não geram respostas imunes.


\chapter{Sistemas imunológicos artificiais}

A modelagem de um problema como um sistema imunológico artificial requer a definição de quatro componentes: codificação, medida de similaridade, seleção e mutação. De maneira geral, o funcionamento do sistema é: uma vez que a codificação e uma medida de similaridade tenham sido escolhidas, o sistema executa a seleção e mutação, ambas baseadas na medida de similaridade, até que o critério de parada seja satisfeito \cite{Aickelin2005}.

\section{Codificação}

A codificação é uma etapa essencial na definição de um Sistema Imunológico Artificial. Ela afeta toda modelagem do sistema, em especial a medida de similaridade, e pode ser responsável pelo seu sucesso ou falha. Os dois principais agentes do sistema imunológico, antígenos e anticorpos, são também os principais elementos que devem ser modelados, e são representados da mesma forma.

Um antígeno é uma parte do objetivo ou solução da aplicação, que pode ser único ou um parte de um conjunto-solução. Os anticorpos são o resto dos dados, e geralmente existem em grande quantidade. Em uma aplicação de detecção de intrusão, por exemplo, o antígeno poderia ser o conjunto de dados que define um tráfego de dados, e os anticorpos, os conjuntos de dados que já foram identificados como legais.

Na maioria dos problemas, a representação desses dois elementos é feita através de um vetor de números ou de características. Cada posição desse vetor representa uma característica da instância. Os tipos mais comuns de dados são números (inteiros ou reais), \emph{strings} e valores binários.

\section{Medida de similaridade}

A medidade de similaridade (também chamada de medida de afinidade, principalmente na área de Sistemas Imunológicos Artificiais) é usada para comparar duas instâncias, e mede o quanto uma é semelhante à outra. É usada principalmente para agrupar instâncias em grupos e nas condições de término (o algoritmo termina quando o modelo descreve as instâncias com uma medida de similaridade satisfatória, que varia de acordo com a aplicação, o tempo de execução e o nível de similaridade necessário).

Uma função de similaridade simples é o número de \emph{bits} que são idênticos nas duas sequências. Por exemplo, para os \emph{strings} (00011) e (00000), a função retornaria 3. Essa função é oposta à distância de Hamming, que mede quantos \emph{bits} devem ter seus valores trocados para tornar as sequências iguais. Em alguns problemas, essa medida não é suficiente, já que ela não tem a noção de continuidade. Uma alternativa é calcular o número de posições iguais contínuas, retornando o maior valor. Assim, o exemplo anterior também receberia o valor 3, mas as sequências (00000) e (01010) receberia 1. Essa diferenciação pode ter ser apropriada ou não, dependendo do problema. Para variáveis não-binárias, existem ainda mais possibilidades de medida de distância, como a distância Euclideana.

Para os problemas de mineração de dados, a aptidão geralmente significa correlação, e uma medida simples de correlação é o coeficiente de correlação de Pearson. A correlação de duas instâncias \emph{u} e \emph{v} é definida como:

\begin{equation}
r=\frac{
    \sum\limits_{i=1}^{n}
        (u_i-\overline{u})
        (v_i-\overline{v})
    }
    {\sqrt{
        \sum\limits_{i=1}^{n}
            (u_i-\overline{u})^2
        \sum\limits_{i=1}^{n}
            (v_i-\overline(v))^2
        }
    }
\end{equation}

Onde \emph{u} e \emph{v} são duas instâncias, \emph{n} é o número de variáveis comuns a u e v, \emph{$u_i$} é o valor da varável i na instância u e \emph{$\overline{u}$} é a média dos valores de todas as variáveis de u (não apenas das variáveis comuns a u e v). A média é modificada para que o valor 0 seja atribuído a instâncias sem nenhuma variável em comum. O resultado são valores de -1 a 1, onde 1 significa forte concordância e -1 forte discordância.

Para algumas aplicações, aptidão pode não ser benéfica, e as instâncias que são mais semelhantes são na verdade descartadas. Esse processo é conhecido como seleção negativa, e é equivalente ao processo que acredita-se que ocorre durante a maturação dos linfócitos B, onde eles aprendem a não identificar os tecidos próprias, para que não se inicie uma reação auto-imune.

A seleção negativa é muito aplicada a sistemas de segurança. Cria-se um ambiente seguro, formado apenas por componentes confiáveis. Na inicialização do sistema, é gerado um grande número de detectores randômicos, que são aplicados aos dados gerados por esse ambiente. Aqueles que identificarem os dados legais são eliminados, restando ao final apenas aqueles que não identificarem. Esses detectores formarão o sistema de detecção, que irá monitorar constantemente o ambiente. Caso algum detector identifique os dados no ambiente, é gerado um alerta de "possível não-próprio".

A otimização da aptidão dos modelos em muitos casos não é realmente o objetivo do sistema se o objetivo é criar generalizações através de um subconjunto dos dados existentes \cite{Hand2001}. Um modelo com grande aptidão pode não se adaptar a novas instâncias que venham a ser analizadas pelo sistema.

\section{Seleção}
\section{Mutação}


\chapter{Detecção de fraude}

A fraude pode ter origem tanto interna quanto externa a uma organização. Por exempo, uma empresa está sujeita a fraude por seus administradores (denominada de alto-nível) ou empregados que não sejam gestores (baixo-nível) \cite{Phua2010}. Em um documento de 2012, a Associação de Investigadores Certificados de Fraude (Association of Certified Fraud Examiners, ACFE) definiu a fraude interna como a exploração ilegal dos recursos e bens de uma empresa, por um empregado, para enriquecimento próprio \cite{ACFE2012}.

Já na fraude externa, os seus autores dividem-se em três perfis: casual, criminal e crime organizado \cite{Phua2010}. Criminosos casuais apresentam comportamento aleatório, transgredindo as leis quando têm oportunidade, tentação ou em períodos de dificuldades financeiras. Por outro lado, indivíduos ou grupos organizados são mais perigosos porque tentam esconder ou dissimular sua verdadeira identidade, além de evoluir seu \emph{modus operandi} com o tempo, tentando burlar os sistemas de detecção e evitar a sua identificação. Assim, é importante levar-se em consideração essa constante interação entre os sistemas de detecção e os fraudadores profissionais. Essas categorias de fraudadores geralmente atuam em um setor específico: as fraudes internas e de seguro são mais frequentemente exploradas por criminosos comuns, enquanto fraudes de cartão de crédito e telecomunicações são vítimas de fraudadores profissionais.

O monitoramento de sistemas com o objetivo de encontrar comportamentos fraudulentos já existia muito antes da utilização dos sistemas computacionais tornarem-se ferramentas comuns. Antes havia um processo denominado auditoria: gerar, armazenar e revisar um registro cronológico de eventos de um sistema \cite{Bace2000}. Os principais objetivos dos sistemas de auditoria são identificar os usuários do sistema, impedir o uso impróprio, e auxiliar na reconstrução de eventos e na estimativa, quantificação e qualificação de danos.

O primeiro trabalho a considerar necessária a auditoria automática de sistemas foi \citet{Anderson1972}. Nesse trabalho, Anderson classifica os riscos e ameaças a sistemas, diferenciando fontes internas e externas, como na figura \ref{fraud:and}. Ele também cita diversos objetivos para um sistema de auditoria:

\begin{enumerate}[a)]
    \item Prover informações suficientes para que o problema possa ser localizado, mas que não exponham detalhes que possibilitem um ataque.
    \item Obter dados de diversas fontes para otimizar o conteúdo do banco de dados.
    \item Discernir uma atividade ``normal'' do sistema, para que se possa detectar abusos interno.
    \item Levar em consideração as estratégias dos invasores no projeto do sistema.
\end{enumerate}

A detecção de fraude é apenas uma das etapas de um sistema chamado de \emph{controle de fraude}. Nesse contexto, a detecção automática ajuda a reduzir o trabalho manual de verificação das instâncias \cite{Phua2010}. O objetivo principal desses sistemas é identificar padrões de transações suspeitas em meio às transações comuns de uma organização. O fraudador pode, por exemplo, contratar um seguro usando informações de outra pessoa ou informações falsas. O sistema procura detectar e impedir a fraude o mais cedo possível.

\renewcommand{\arraystretch}{1.5}
\begin{table}[h!]
    \caption{Matriz de ameaças}
    \label{fraud:and}
    \centering
    \begin{tabular}{c p{4cm}|>{\centering\arraybackslash}p{4cm}|>{\centering\arraybackslash}p{4cm}|}
        \cline{3-4}
        & & Uso não autorizado dos dados/programa & Uso autorizado dos dados/programa \\
        \hhline{~---}
        \multicolumn{0}{c|}{} & Uso não autorizado do computador & Invasão externa & \cellcolor{gray!90} \\
        \cline{2-4}
        \multicolumn{0}{c|}{} & Uso autorizado do computador & Invasão interna & Abuso de poder \\
        \cline{2-4}
    \end{tabular}
    \\ Fonte: \cite{Anderson1972}.
\end{table}

Geralmente não é possível ter absoluta certeza sobre a legitimidade das transações de um negócio a partir dos dados disponíveis. Não seria possível verificar todas as entidades com as quais uma empresa mantém relações, que em algumas empresas podem ser milhares. Assim, não existe uma técnica infalível para a detecção de fraudes. Isso não significa que elas não possam ser detectadas. A melhor alterantiva, na prática, é uma busca por possíveis evidências de fraude nos dados disponíveis. Métodos matemáticos e estatísticos são utilizados para comparar um banco de dados de transações existentes com as novas, com o objetivo de encontrar evidências de possíveis fraudes. Mesmo assim, geralmente há um processo de análise e revisão caso a caso por um especialista.

Ainda, segundo a pesquisa apresentada no mesmo trabalho, o motor analítico desse tipo de sistema pode ser composto de um ou mais métodos, tais como: Sistemas Imunológicos Artificiais, inteligência artificial, auditoria, bancos de dados, computação distribuída e paralela, econometria, sistemas especialistas, lógica nebulosa, algoritmos genéticos, aprendizagem de máquina, redes neurais, reconhecimento de padrões, estatística, visualização, entre outros.

Dois conceitos relacionados à detecção em geral são falsos positivos e falsos negativos. Falsos positivos são instâncias erroneamente classificadas, por exemplo, uma transação comum que é classificada como fraudulenta. Falsos negativos são o oposto: uma transação fraudulenta que é classificada como comum. O número de falsos positivos aumenta o trabalho desnecessário na fase de revisão, enquanto os falsos negativos reduzem a eficácia da detecção.

A redução dos falsos positivos é um dos objetivos principais de um sistema de detecção. Os falsos negativos, no entanto, são quase impossíveis de serem eliminados completamente, em qualquer sistema de detecção \cite{Michie1994}. Mesmo um sistema capaz de reconhecer sinais de fraudes existentes não é suficiente para um ambiente real. Fraudadores tentam constantemente superar os sistemas de detecção, evoluindo o seu \emph{modus operandi} com o tempo, novos métodos são criados, novos fraudadores entram em atividade. Um sistema que almeje deter o maior número possível deve ser constantemente atualizado para se adaptar às mudanças de um ambiente tão dinâmico.

Dependendo do domínio da organização, para que um sistema possa prever fraudes com antecedência suficiente para que elas sejam evitadas, ele deve monitorar constantemente as novas transações em andamento. Esse fator reduz a utilização de sistemas que necessitam de um longo tempo de treinamento ou de análise. O ideal seria que ele estivesse em constante execução, analisando as transações conforme elas ocorrem. Para aplicações grandes ou descentralizadas, pode ser muito difícil conseguir isso sem afetar a performance geral das transações.

Para cada domínio, tipos diferentes de fraude podem existir, e mais de um tipo pode ocorrer simultaneamente, sem uma ordem definida.

\section{Dados}
\label{fraud:data}

Existem alguns conceitos sobre as bases de dados que são característicos da área de mineração de dados e aprendizagem de máquina. O número de instâncias na base é chamado de número de \emph{amostras} e o número de atributos de cada instância é chamado de número de \emph{características} (em inglês, \emph{samples} e \emph{features}).

Os atributos das instâncias em um banco de dados usado para detecção de fraude geralmente limitam-se a valores binários, numéricos, categóricos ou uma mistura desses três. Os atributos específicos usados geralmente são semelhantes. Aplicações de seguro utilizam o histórico do cliente: tempo de contrato, histórico e total de pagamentos, lucro anual valor médio depositado em conta bancária. Para fraudes de crédito, utilizam-se informações sobre as transações: valor, data, localização geográfica, conta de destino, tempo de conta, etc. Fraudes em seguros de automóveis utilizam valores binários para atributos como acidente e tratamento hospitalar, além de dados do motorista, custo, tipo de ferimento, etc.

O número de instâncias positivas nas bases de dados de fraude é geralmente muito reduzido: as fraudes representam um percentual muito pequeno em relação ao número de transações legítimas de uma organização, geralmente menor do que 20\%. Métodos de detecção de fraude nunca são perfeitos: deve existir um mecanismo para lidar com as fraudes que não são identificadas a tempo de serem impedidas.

A obtenção de bancos de dados reais para teste é difícil, já que empresas e organizações, por razões legais e competitivas não disponibilizam informações desse tipo. Assim, é difícil encontrar bases de dados públicos para que os testes sejam realizados. Outra desvantagem comumente encontrada nesses bancos de dados é o fato dos dados estarem alterados, para a preservação da confidencialidade dos clientes das empresas fornecedoras. Apesar dos dados ainda poderem ser utilizados normalmente em ambientes de teste, nao é possível, como seria caso se tivesse acesso aos dados originais, derivar regras de classificiação dos resultados dos testes. Por exemplo, observando os resultados, seria possível perceber que um valor alto em um atributo leva a instância ser classificada como fraudulenta na maioria dos casos. Em uma base de dados esse atributo teria uma descrição informativa, mas isso é perdido em uma base alterada, onde ele seria descrito por um identificador sem significado, como ``atributo 5''.

Uma alternativa é a criação de um banco de dados artificial, inspirado em dados reais. A eficácia dessa técnica é limitada pela capacidade do criador do banco em prever o maior número de casos possíveis, o que geralmente é muito inferior à variedade das situações reais. Mesmo assim, essa técnica é comumente empregada na fase de concepção e teste, devido à dificuldade de obtenção de dados.

\section{Implementação}

A maioria dos sistemas de detecção de fraude opera usando listas negras (\emph{black-lists}) de dados (transações, contas, etc.), que são comparadas com as nova instâncias. Algumas utilizam regras fixas para a classificação. A figura \ref{fig:fraud_data} mostra a organização dos dados nesses sistemas. Uma parte do banco de dados é usada para o treinamento. Esses são os dados onde o sistema aprenderá os padrões e regras utilizados para a detecção. O restante das instâncias é usada para a avaliação do treinamento. Também é comum que haja bancos de dados distintos para o treinamento e avaliação.

\begin{figure}[h!]
    \centering
    \caption{Dados para a análise}
    \label{fig:fraud_data}
    \includegraphics[width=0.75\textwidth]{img/fraud-data.png}
    \\ Fonte: \cite{Phua2010}.
\end{figure}

A maioria dos estudos relacionados à detecção de fraude considera a detecção de \emph{outliers} como uma ferramenta principal de detecção \cite{Aral2011}. Existem muitos métodos aplicados à detecção de fraude: auditoria, sistemas especialistas, lógica nebulosa (\emph{fuzzy}), redes neurais, reconhecimento de padrões, árvores de decisão, regressão, etc \cite{Huang2010}. Considerando os dados dividos conforme a figura \ref{fig:fraud_data}, as duas técnicas mais usadas são:

\begin{enumerate}[a)]
\item Dados para treinamento classificados (A + B + C + D) processados por um algoritmo supervisionado.
\item Instâncias legais (C) processadas por um algoritmo semi-supervisionado.
\end{enumerate}

Algoritmos supervisionados examinam as instâncias previamente classificadas (A + B + C + D) para identificar matematicamente os padrões presentes nas classificadas como fraudulentas. Os algoritmos mais utilizados nessa categoria são as redes neurais. Outros algoritmos incluem máquinas de vetores de suporte (\emph{support vector machine}, SVM), árvores de decisão e raciocínio baseado em casos. Para aumentar a eficácia dos métodos supervisionados, esses algoritmos podem ser aplicados em sequência. Também podem ser combinados resultados de bancos de dados distintos.

Algoritmos não-supervisionados atuam sobre dados não classificados (A + C + E + F), e o seu objetivo é agrupar os dados em padrões, para que estes sejam mais facilmente analisados, combinando a detecção humana e a computação da máquina. Exemplos desses algoritmos são redes neurais não supervisionadas, análise de ligações (\emph{link analysis}) e mineração de grafos (\emph{graph mining}).

A combinação de dois ou mais algoritmos supervisionados, ou de algoritmos supervisionados e não-supervisionados, chamados de algoritmos híbridos, é uma grande área de pesquisa. Também são utilizados algoritmos supervisionados em bancos de dados que contêm apenas instâncias legais (C). Regras são geradas e testadas na base de dados, e aquelas que identificam padrões nesses dados são descartadas. Esses algoritmos são chamados de semi-supervisionados, porque apesar de não fazerem distinção entre dados legais e ilegais, os dados ainda necessitam ser classificados para que sejam utilizados no seu treinamento.

Autores como Phua fazem muitas críticas ao uso de dados previamente classificados para o treinamento dos sistemas \cite{Phua2010}. A classificação atrasa o processo de detecção, aumenta o tempo de reação a novos tipos de fraudes e pode ser cara e difícil de se obter. Pode ainda ser incorreta, tendenciosa e expor dados sigilosos, dependendo do tipo de aplicação. Assim, instâncias de treinamento e avaliação (A + C + E + F, sem classificações) devem ser combinadas e processadas por um algoritmo não-supervisionado, detectando regras, pontuações ou anomalias visuais nos dados avaliados.

Um sistema que detecte e reporte uma fraude muito tempo depois de ela ter ocorrido permite que o fraudador consiga causar um dano substancial. Em geral, esse tempo de resposta de um sistema a partir do momento em que uma fraude é concretizada até a sua detecção é crucial para a eficácia do sistema em de fato proteger o ambiente em que é inserido.

Os sistemas de detecção podem ser divididos em dois tipos gerais, denominados \emph{on-line} e \emph{off-line}, enquanto alguns incorporam características dos dois modelos e outros têm um processo distinto para ambos, que interagem entre si para gerar o resultado final.

\iffalse

\cite{Huang2010}.

\fi

\section{Critérios de avaliação}
\label{sec:fraud_criteria}

Phua e outros autores listam as diversas medidas de performance utilizadas em trabalhos recentes na área de detecção de fraude \cite{Phua2010}. Os métodos tradicionais de medição, como a taxa de positivos (número de fraudes detectadas corretamente dividido pelo número de fraudes verdadeiras) e a precisão a um determinado limite (número de instâncias classificadas corretamente, dividido pelo número total de instâncias) foram abandonados pelos trabalhos recentes, devido à natureza peculiar da detecção de fraude.

A razão para isso é que o custo das classificações errôneas, no caso da detecção de fraude, são irregulares, incertos, variam de exemplo a exemplo e podem variar conforme o tempo. Falsos negativos são geralmente mais custosos do que falsos positivos. Um falso positivo geralmente leva apenas a uma verificação desnecessária por um especialista. Um falso negativo, no entanto, acarreta em uma fraude que não é detectada pelo sistema e não será reportada, deixando o fraudador impune. Mesmo assim, muitos dos sistemas de detecção comerciais, como a maioria dos departamentos governamentais que atuam na área de detecção de fraude, utilizam valores monetários como medida de avaliação \cite{Phua2010}.

Em relação à comparação da eficácia dos métodos, destacam-se os seguintes critérios:

\begin{enumerate}[a)]
    \item Similaridade de uma instância com os exemplos de fraude conhecidos divida pela dissimilaridade com exemplos legais.
    \item Análise da curva ROC (\emph{Receiver Operating Characteristic}, Característica de Operação do Receptor): a taxa de verdadeiros positivos para falsos positivos.
    \item Análise da área sob a curva ROC, que mede a probabilidade de uma instância positiva ser classificada como "mais positiva" do que uma instância negativa.
    \item Entropia cruzada: mede a diferença entre o valor atribuído na classificação de uma instância e o valor real atribuído àquela instância nos dados de teste.
    \item Escore Brier: o erro quadrático médio, ou seja, a média do somatório dos quadrados da diferença entre a classificação de uma instância e a classificação real.
    \item Alguns trabalhos adotam um critério peculiar: as instâncias são classificadas de acordo com o seu valor monetário. Esse valor pode ser tanto explícito quanto baseado em modelos financeiros.
\end{enumerate}

Em relação à eficiência dos métodos, destacam-se os seguintes critérios:

\begin{enumerate}[a)]
    \item Velocidade de detecção (tempo até o alarme).
    \item Número de tipos ou estilos de fraude detectados.
    \item Processo de de detecção em tempo real ou em lote.
\end{enumerate}

\subsection{\emph{Cross-validation}}

Quando um modelo é criado à partir de um conjunto de dados e os testes são executados sobre esse mesmo conjunto, é muito provável que eles apresentem um bom resultado. No entanto, isso não garante que o modelo pode ser aplicado com a mesma eficácia sobre novos dados, o que geralmente é o objetivo dos sistemas de detecção.

Para que seja possível avaliar se um modelo pode prever novos dados após o treinamento, a base de dados deve ser dividida em duas partes: um conjunto de treinamento e um conjunto de teste. O modelo é criado utilizando apenas os dados de treinamento e avaliado utilizando os dados de teste.

Caso o resultado dos testes de um modelo quando aplicado aos mesmos dados utilizados no treinamento seja bom, mas o resultado quando aplicado aos dados de teste sejam ruims, diz-se que ocorre \emph{overfitting}: o modelo ajusta-se tanto aos dados de treinamento que não consegue criar regras genéricas o suficiente para identificar amostras semelhantes mas não exatamente iguais.

Há ainda a possibilidade do modelo não gerar um resultado bom nem mesmo para os dados de treinamento. Nesse caso, diz-se que ocorre \emph{underfitting}: o modelo não é capaz de criar regras que se ajustem aos dados de treinamento.

No entanto, o resultado dessa separação pode ser altamente dependente da separação dos dados. Separar os dados de forma diferente geralmente altera o resultado dos testes. Para evitar essa dependência, pode ser utilizado um processo denominado \emph{cross-validation}:

\begin{enumerate}[a)]
    \item Dividir aleatoriamente os dados em dois conjuntos, treinamento e testes.
    \item Criar o modelo com base no conjunto de dados de treinamento.
    \item Testar o modelo utilizando o conjunto de dados de testes.
    \item Repetir os passos a) a c).
    \item Calcular a média e o erro padrão da média dos resultados de cada iteração.
\end{enumerate}

\subsection{Avaliação de classificadores}

Classificadores binários são aqueles que, para um dado conjunto de valores de entrada, geram como valor de saída um valor Booleano: verdadeiro ou falso, positivo ou negativo. Os sistemas de detecção de fraude se situam, na maioria dos casos, nessa categoria: os dados referentes a uma instância passam pelo sistema de detecção e são classificados como legítimos ou fraudulentos \cite{Bewick2004}. Sistemas mais complexos podem gerar saídas com mais informações sobre a instância, como o grau de certeza da previsão ou o grau de semelhança entre a instância e os dados da base.

A avaliação dos classificadores binários é feita com base em uma \emph{tabela de confusão}, conforme mostrado na tabela \ref{fraud:confusion}. Nessa tabela são listados o número de valores (ou porcentagem) para cada combinação de saída do sistema e a classificação real. Instâncias corretamente classificadas ocupam as colunas Positivo e Negativo, enquanto instâncias cujo classificação difere da realidade ocupam as colunas ``Falsos positivos'' e ``Falsos negativos''.

\renewcommand{\arraystretch}{1.5}
\begin{table}[h!]
    \vspace{1cm}
    \caption{Tabela de confusão}
    \centering
    \begin{tabular}{c l c c}
        & & \multicolumn{2}{c}{\textbf{Dados reais}} \\
        \multirow{3}{5mm}{\begin{sideways}\parbox{20mm}{\textbf{Saída}}\end{sideways}} & \multicolumn{1}{c|}{} & Positivo & Negativo \\
        \cline{2-4}
        & \multicolumn{1}{c|}{Positivo} & Positivo & Falso positivo\\
        & \multicolumn{1}{c|}{Negativo} & Falso negativo & Negativo\\
    \end{tabular}
    \label{fraud:confusion}
    \vspace{1cm}
\end{table}

Como exemplo, inspirado em \citet{Bewick2004}, é apresentado nessa seção um sistema de detecção de fraude, que considera como único atributo das instâncias o número de ocorrências de um determinado valor. Nesse ambiente simplificado, a instância é considerada fraudulenta caso um limiar de ocorrências seja ultrapassado.

Esse sistema de testes utiliza como único critério de avaliação o número de ocorrências (segunda coluna). Um \emph{limiar de classificação} é definido: um parâmetro que o sistema de detecção utilizará para guiar a previsão da detecção. Se o limiar de classificação fosse 1, todas as instâncias que tivessem mais de uma ocorrência (nos valores de exemplo, as instâncias 0 e 2) seriam consideradas fraudulentas.

Um exemplo do resultado de um teste nesse sistema é mostrado na tabela \ref{fraud:ex}. As colunas sob ``Dados reais'' mostram o número de instâncias fraudulentas e legítimas; as linhas sob ``Saída'' mostram a classificação gerada pelo sistema. A partir desses dados, a avaliação dos resultados é feita usando dois conceitos: sensibilidade e especificidade.

\renewcommand{\arraystretch}{1.5}
\begin{table}[h!]
    \centering
    \caption{Exemplo de tabela de confusão}
    \label{fraud:ex}
    \begin{tabular}{c l c c c}
        & & \multicolumn{2}{c}{\textbf{Dados reais}} \\
        \multirow{3}{5mm}{\begin{sideways}\parbox{20mm}{\textbf{Saída}}\end{sideways}} & \multicolumn{1}{c|}{} & Fraude & Legítimo & \multicolumn{1}{|c}{Total} \\
        \cline{2-5}
        & \multicolumn{1}{c|}{Fraude}   & 300 & 200   & \multicolumn{1}{|c}{500}  \\
        & \multicolumn{1}{c|}{Legítimo} & 100 & 1000  & \multicolumn{1}{|c}{1100} \\
        \cline{2-5}
        & \multicolumn{1}{c|}{Total}    & 400 & 1200  & \multicolumn{1}{|c}{1600} \\
    \end{tabular}
\end{table}

\subsection{Sensibilidade e especificidade}

\emph{Sensibilidade} refere-se à proporção de instâncias corretamente classificadas como positivas. \emph{Especificidade} refere-se à proporção de instâncias corretamente classificadas como negativas. Tomando como exemplo os dados da tabela \ref{fraud:ex}, a sensibilidade é igual a 300 / 400 = 0.75; e a especificidade é igual a 1000 / 1200 = 0.8333. Ou, visto de outra maneira, 75\% das fraudes foram realmente classificadas como fraudes; e 83.33\% das instâncias legítimas foram classificadas como legítimas.

Apenas a consideração desses dois valores pode levar a uma correta avaliação dos resultados do teste, principalmente nos domínios da detecção de fraude. Uma alta sensibilidade não necessariamente implica em uma alta especificidade, e vice versa. Dessas duas informações são derivados os \emph{valores preditivos positivos e negativos}.

\subsection{Valores preditivos}

O \emph{valor preditivo positivo} é a chance de uma instância ser realmente uma fraude caso seja classificada como tal pelo sistema. O \emph{valor preditivo negativo} é a chance de uma instância ser realmente legítima caso seja classificada como tal pelo sistema. Para os dados de exemplo, o valor preditivo positivo é 300 / 500 = 0.6; e o valor preditivo negativo é 1000 / 1100 = 0.9091. Ou, visto de outra maneira, 60\% das instâncias que foram classificadas como fraude eram realmente fraudes, e 90.91\% das instâncias classificadas como legítimas eram realmente legítimas.

Esses dois valores são o oposto da sensibilidade e especificidade, respectivamente. Enquanto os valores preditivos dão uma avaliação direta sobre os resultados dos testes, a sensibilidade e especificidade não são afetadas pela proporção dos valores nas instâncias, ou seja, não são alteradas quando há uma alteração na proporção de fraudes.

\subsection{Taxas de verossimilhança}

A sensibilidade e especificidade tornam-se ainda mais úteis quando combinadas, gerando as taxas de verossimilhança. A taxa de verossimilhança de um resultado positivo (LR\textsuperscript{+}) é a razão entre a probabilidade de um resultado positivo no teste caso a instância seja realmente positiva (coluna "positivo" da tabela de confusão) e a probabilidade de um resultado positivo no teste caso a instância seja na verdade negativa:

\begin{equation}
    \vspace{2mm}
    LR^{+}=\frac{sensibilidade}{1 - especificidade}
    \vspace{2mm}
\end{equation}

No exemplo, LR\textsuperscript{+} = 0.75 / (1 - 0.8333) = 4.4991. Isso significa que um resultado positivo nos testes é 4.4991 vezes mais provável para instâncias que são realmente positivas do que para aquelas que não são.

De maneira, similar, a taxa de verossimilhança negativa (LR\textsuperscript{-}) é a razão entre a probabilidade de um resultado negativo no teste caso a instância seja realmente negativa (coluna "negativo" da tabela de confusão) e a probabilidade de um resultado negativo no teste caso a instância seja na verdade positiva:

\begin{equation}
    \vspace{2mm}
    LR^{-}=\frac{1 - sensibilidade}{especificidade}
    \vspace{2mm}
\end{equation}

No exemplo, LR\textsuperscript{-} = (1 - 0.75) / 0.8333 = 0.3, o que significa que um resultado negativo no teste é 0.3 vezes mais provável para instâncias que são realmente negativas do que para aquelas que não são.

Desses valores, pode-se atestar a utilidade do método para classificação. Uma alta taxa de verossimilhança positiva indica que o teste é útil para verificar se uma instância é positiva, enquanto uma alta taxa de verossimilhança negativa é útil para verificar se uma instância é negativa. Assim como os valores preditivos, essas taxas são sensíveis à predisposição dos dados na base de dados.

\subsection{Índice de Youden e ROC}

Os dados de teste do exemplo mostrados nas tabelas até agora consideraram apenas um valor como limiar de classificação. Mudanças nesse limiar afetam a precisão dos testes: caso o limiar seja seja igual a -1, todas as instâncias são classificadas como fraudulentas, já que todas possuem como número de ocorrências um número maior ou igual a zero. Por outro lado, caso o limiar seja um número maior do que qualquer uma das instâncias, todas serão classificadas como legítimas. Em sistemas reais, é comum que sejam testados diversos valores para esse limiar sobre uma mesma base de dados, para que o melhor seja identificado e usado no sistema final. A tabela \ref{fraud:youden} é um exemplo de resultados para esse tipo de teste.

\begin{table}[h!]
    \caption{Exemplo de resultados para testes comparativos de limiares}
    \label{fraud:youden}
    \centering
    \begin{tabular}{l c c c c c}
        \hline
        Limiar & Fraudes & Legítimos & Sensibilidade & Especificidade & Índice de Youden \\
        \hline
        0     & 400 &    0 & 1      & 0      & 0      \\
        1     & 395 &  400 & 0.9875 & 0.3333 & 0.3208  \\
        2     & 380 &  600 & 0.95   & 0.5    & 0.45   \\
        3     & 375 &  900 & 0.9375 & 0.75   & 0.6875 \\
        4     & 200 & 1000 & 0.5    & 0.8333 & 0.3333 \\
        99    &   0 & 1200 & 0      & 1      & 0      \\
        \hline
        Total & 400 & 1200 &    -   &    -   &    -   \\
        \hline
    \end{tabular}
    \\ Fonte: inspirado em \citet{Bewick2004}.
\end{table}

Uma medida para a avaliação dos diferentes limiares pode ser a sensibilidade e especificidade. O índice de Youden (J), que é uma derivação desses valores, é uma medida apropriada nesses casos (eq. \ref{fraud:yindex}). Esse índice é um valor normalizado na faixa [0, 1], onde 1 significa um teste perfeito, classificando todas as instâncias corretamente; e 0 significa um teste sem valor nenhum. A tabela \ref{fraud:yindex} mostra os valores para os resultados com diversos limiares. É importante notar como valores extremos maximizam a sensibilidade e a especificidade, mas apenas aqueles limiares que maximizam os dois valores recebem índices significativos.

\begin{equation}
    \vspace{2mm}
    J = sensibilidade + especificidade - 1
    \label{fraud:yindex}
    \vspace{2mm}
\end{equation}

Uma característica importante do índice de Youden é que ele estabelece uma equivalência na relevância dos falsos positivos e negativos. Quando um é mais relevante que o outro, devido à peculiaridades nos dados ou no domínio do problema, ele não é apropriado. Nesses casos, pode-se atribuir pesos diferentes a ambos os valores. Um exemplo de adaptação do cálculo do índice de Youden é mostrado abaixo. Nessa fórmula, os valores ainda são mantidos na mesma faixa, mas falsos positivos recebem o dobro da significância.

\begin{equation}
    \vspace{2mm}
    J = 0.75 * sensibilidade + 0.25 * especificidade
    \vspace{2mm}
\end{equation}

Como pode-se observar na tabela \ref{fraud:yindex}, alterações no valor usado na classificação alteram o número de instâncias classificadas como positivas e negativas. Geralmente os resultados são distribuídos começando com um grande valor de sensibilidade e um valor baixo de especificidade; esses valores convergem até um máximo global; e então terminam em com um valor baixo de sensibilidade e um grande valor de especificidade. A variação desses três valores, e a interação entre eles, é mostrada no gráfico da figura \ref{fraud:threshold}.

\begin{figure}[h!]
    \centering
    \caption{Variação dos valores de avaliação conforme o limiar de detecção}
    \label{fraud:threshold}
    \includegraphics[scale=0.5]{img/threshold.png}
\end{figure}

No lado esquerdo, onde o limiar é muito baixo, a sensibilidade é 1 e a especificidade, zero; enquanto no lado direito, onde o limiar é muito alto, ocorre o oposto: a sensibilidade é zero e a sensibilidade, 1. Conforme o limiar aumenta (começando do lado esquerdo e caminhando para a direita), ocorreu uma diminuição na especificidade (algumas instâncias fraudulentas começam a ser classificadas como legítimas). No entanto, há um aumento muito maior na especificidade (instâncias legítimas começam a ser classificadas corretamente). É possível ver como o índice de Youden mostra a real proporção entre esses dois valores. Uma forma mais sucinta para apresentar esse tipo de informação é através de um gráfico dos valores de sensibilidade e 1 - especificidade, que é denominado ROC (característica operativa do receptor, \emph{receiver operating characteristic}), conforme mostrado na figura \ref{fraud:roc}.

\begin{figure}[h!]
    \centering
    \caption{Gráfico da curva ROC dos valores de exemplo}
    \label{fraud:roc}
    \includegraphics[scale=0.5]{img/roc.png}
\end{figure}

No gráfico é visualmente aparente que o valor 3 como limiar de classificação oferece o melhor balanceamento entre sensibilidade e especificidade. Também é possível perceber como as alterações no limiar influenciam a distribuição desses dois valores. Um teste perfeito, que classificasse todas as instâncias corretamente, teria ambos os valores iguais a 1, tendo portanto um ponto no canto superior esquerdo do gráfico. Os pontos mais próximos dessa localização são aqueles que melhor classificam as instâncias.

Caso as instâncias fossem classificadas aleatoriamente, tendo chances iguais de serem classificadas tanto positivas quanto negativas, a sensibilidade e a especificidade seria ambas 0.5, e o teste não teria qualquer valor. Essa situação é representada pela linha diagonal entre os pontos (0, 1) e (1, 0). Essa linha é outro indicador visual útil: testes abaixo dela (na direção do canto inferior direito) têm muito pouco valor; testes próximos dela têm valor mediano; e testes acima (na direção do canto superior esquerdo) têm grande valor.

Uma das maneiras de quantificar a (dar um valor à) validade de um atributo como variável de diagnóstico é através do cálculo dá área sob a curva ROC (denominada AUROC, \emph{area under the ROC curve}). O teste ideal citado acima teria área igual a 1, enquanto testes completamente aleatórios teriam área 0.5. Testes reais normalmente situam-se entre esses dois valores, com valores mais próximos de 1 representando testes mais precisos.

O cálculo da área pode ser feito somando-se a área dos trapézios formados pela curva. Tomando dois pontos da curva, (0.1667, 0.5) e (0.3125, 0.9375), o trapézio formado por esses pontos e os pontos (0.1667, 0.0) e (0.3125, 0.0) é igual a (0.9375 - 0.5) x (0.3125 + 0.1667)/2 = 0.1048. Aplicando essa regra aos outros pontos da curva, obtêm-se uma área total de 0.8161, ou seja, uma instância fraudulenta tem 81.61\% de ter um número de ocorrências de inadimplência maior do que uma instância legítima.

\subsection{Análise de resultados}

Uma vez que possa ser quantificada, a capacidade de diagnóstico de uma variávei pode ser comparada com outras, simplesmente comparando-se as suas curvas ROC e as áreas sob essas curvas. Variáveis com uma maior área sob a curva geram diagnósticos mais confiáveis. O formato da curva também pode servir como instrumento de análise: uma variável pode ter comportamentos preferíveis caso as condições de teste sejam diferenciadas, ou para filtrar casos específicos. Por exemplo, uma variável que, para valores muito baixos de sensibilidade, apresenta uma alta taxa de especificidade pode ser usada quando quer-se favorecer a especificidade.

A área sob a curva ROC é um método útil para a medição da precisão de diagnósticos, além da comparação de performance entre diferentes testes. No entanto, ela não deve ser tomada como verdade absoluta. A sensibilidde e a especificidade podem manter-se fixas em um ambiente de testes, mas podem variar conforme as características da populaçõ analisada.

Uma consideração importante quando analisa-se o desempenho de dois ou mais sistemas é que muito difícil (ou até mesmo impossível) excluir-se todos os fatores externos, resultando em uma análise completamente imparcial. Nesse caso específico, o resultado dos testes em cada algoritmo é fortemente influenciado pelas características dos dados na base de dados, embora os métodos de análise dos resultados procuram minimizar essa influência.

\iffalse

Uma alternativa é atribuir uma pontuação a cada instância de acordo com uma razão entre a similaridade com casos conhecidos de fraude e a dissimilaridade com casos legais.

\citet{Aral2011} utilizam o número de falsos positivos, falsos negativos e verdadeiros positivos como medida de performance do seu sistema.

\section{Trabalhos relacionados}

Colocar na introdução

O sistema detecção de fraude no comércio eletrônico apresentado em \citet{Huang2010} é um exemplo de sistema que combina dois dois processos de diferentes granularidades que interagem entre si.

\citet{Aral2011} apresentam um sistema de detecção de fraude em prescrições médicas que calcula um fator de risco associado a cada prescrição. Matrizes de incidências são geradas através de uma medida de distância entre valores cruzados (\emph{cross-features}) de uma base de conhecimento. Dessas matrizes são geradas matrizes de riscos, e cada instância da base de dados cujo risco for maior que um limiar é relatada como uma possível fraude. Os autores indicam que o raciocínio por trás desse modelo é de que os padrões no comportamento fraudulento são \emph{outliers} quando considerados no contexto do \emph{dataset} como um todo. Os limiares são configuráveis, permitindo ao usuário um controle sobre o número de falsos positivos gerados.

\fi

\chapter{Proposta}

\section{Objetivos}

Conforme apresentado nos capítulos anteriores, a modelagem de um sistema de detecção de fraude com a utilização de técnicas inspiradas no sistema imunológico natural oferece diversas vantagens. Esse trabalho busca identificar exatamente \emph{quanto} esses modelos podem aperfeiçoar os métodos existentes. O método utilizado é a comparação dos resultados da classificação de algoritmos algoritmos tradicionais e Sistemas Imunológicos Artificiais sobre uma mesma base de dados.

\subsection{Bases de dados}

Para os testes, serão utilizadas duas bases contendo informações de contas de cartões de crédito. Essas bases de dados estão disponíveis publicamente, e fazem parte de um projeto chamado StatLog. Esse projeto foi concebido para testar diversos métodos de classificação em problemas grandes e comercialmente relevantes, comparando os resultados e determinando o quanto essas diferentes técnicas atendiam as necessidades da indústria. Conforme a própria descrição do projeto, os objetivos do projeto eram três:

\begin{enumerate}[a)]
    \item Possibilitar medidas críticas de desempenho para procedimentos de classificação disponíveis
    \item Indicar a natureza e escopo dos desenvolvimentos futuros necessários para que os métodos atendam as necessidades e expectativas dos usuários
    \item Indicar as direções mais promissoras de desenvolvimento para abordagens comercialmente imaturas.
\end{enumerate}

Os atributos do primeiro \emph{dataset} são descritos na listagem \ref{lst:german_dataset} \footnote{Na descrição dos atributos, DM significa \emph{deutsche mark} (marco alemão), a moeda corrente na Alemanha na época da coleta dos dados. Para efeito de comparação, o Banco Central Europeu estipulou a conversão irrevogável do marco alemão, a partir de 1º de janeiro de 1999, em DM 1.95583 = \euro 1 (http://www.ecb.int/press/pr/date/1998/html/pr981231\textunderscore{}2.en.html).}.

\begin{lstlisting}[caption=Atributos do primeiro dataset,label=lst:german_dataset]
    Atributo 1: (qualitativo)
    Situação da conta corrente existente
    A11 : ... < 0 DM
    A12 : 0 <= ... < 200 DM
    A13 : ... >= 200 DM
    A14 : sem conta corrente

    Atributo 2: (numérico)
    Duração em meses

    Atributo 3: (qualitativo)
    Histórico de crédito
    A30 : nenhum crédito retirado / todos os créditos pagos apropriadamente
    A31 : todos os créditos nesse banco pagos apropriadamente
    A32 : créditos existentes pagos apropriadamente até agora
    A33 : atraso no pagamento no passado
    A34 : conta crítica / outros créditos existentes (não nesse banco)

    Atributo 4: (qualitativo)
    Propósito
    A40 : carro (novo)
    A41 : carro (usado)
    A42 : móveis/equipamento
    A43 : rádio/televisão
    A44 : eletrodomésico
    A45 : reparos
    A46 : educação
    A47 : (férias - não existe no dataset)
    A48 : reciclagem profissional
    A49 : negócios
    A410 : outros

    Atributo 5: (numérico)
    Quantidade de crédito

    Atributo 6: (qualitativo)
    Poupança
    A61 : ... < 100 DM
    A62 : 100 <= ... < 500 DM
    A63 : 500 <= ... < 1000 DM
    A64 : .. >= 1000 DM
    A65 : desconhecido / sem poupança

    Atributo 7: (qualitativo)
    Emprego atual desde
    A71 : desempregado
    A72 : ... < 1 ano
    A73 : 1 <= ... < 4 anos
    A74 : 4 <= ... < 7 anos
    A75 : .. >= 7 anos

    Atributo 8: (numérico)
    Taxa de parcelamento em porcentagem do rendimento disponível

    Atributo 9: (qualitativo)
    Estado civil e sexo
    A91 : masculino : divorciado/separado
    A92 : feminino : divorciada/separada/casada
    A93 : masculino : solteiro
    A94 : masculino : casado/viúvo
    A95 : feminino : solteira

    Atributo 10: (qualitativo)
    Outros devedores / fiadores
    A101 : nenhum
    A102 : devedor solidário
    A103 : fiador

    Atributo 11: (numérico)
    Residência atual desde

    Atributo 12: (qualitativo)
    Propriedade
    A121 : imóvel
    A122 : se não A121 : financiamento / seguro de vida
    A123 : se não A121/A122 : carro ou outro, não incluso no atributo 6
    A124 : desconhecido / sem propriedade

    Atributo 13: (numérico)
    Idade em anos

    Atributo 14: (qualitativo)
    Outros planos de parcelamento
    A141 : banco
    A142 : lojas
    A143 : nenhum

    Atributo 15: (qualitativo)
    Residência
    A151 : alugada
    A152 : própria
    A153 : gratuita

    Atributo 16: (numérico)
    Número de créditos existentes nesse banco

    Atributo 17: (qualitativo)
    Emprego
    A171 : desempregado / sem proficiência / não-doméstico
    A172 : sem proficiência / doméstico
    A173 : proficiente / funcionário público
    A174 : administrador / auto-empregado / empregado altamente qualificado / oficial

    Atributo 18: (numérico)
    Número de dependentes

    Atributo 19: (qualitativo)
    Telefone
    A191 : nenhum
    A192 : sim, registrado sob o nom do consumidor

    Atributo 20: (qualitativo)
    Trabalhador estrangeiro
    A201 : sim
    A202 : não
\end{lstlisting}

\section{Arquitetura}

\section{Critérios}

Uma consideração importante quando analisa-se o desempenho de dois ou mais sistemas é que muito difícil (ou até mesmo impossível) excluir-se todos os fatores externos, resultando em uma análise completamente imparcial. Nesse caso específico, o resultado dos testes em cada algoritmo é fortemente influenciado pelas características dos dados na base de dados, embora os métodos de análise dos resultados procuram minimizar essa influência (conforme apresentado na seção \ref{sec:fraud_criteria}).

\section{Resultados}

\chapter{Conclusão}

\section{Síntese}

\section{TCC II}


% Bibliografia.
\bibliography{tcc}
\bibliographystyle{abnt}

\end{document}
