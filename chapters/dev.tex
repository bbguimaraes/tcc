\chapter{Desenvolvimento}

\section{Preparação dos dados}

Os dois conjuntos de dados se encontravam em formato de texto, com os atributos separados por espaços e as instâncias separadas pelo caractere nova linha. A listagem \ref{lst:dev_data} mostra as primeiras cinco instâncias do \emph{Cr.Ger} no formato do arquivo original (a primeira coluna representa o número da linha, e não faz parte dos dados).

\vspace{0.5cm}
\begin{lstlisting}[caption=Formato original dos dados (\emph{Cr.Ger}), label=lst:dev_data]
A11 6 A34 A43 1169 A65 A75 4 A93 A101 4 A121 67 A143 A152 2 A173 1 A192 A201 1
A12 48 A32 A43 5951 A61 A73 2 A92 A101 2 A121 22 A143 A152 1 A173 1 A191 A201 2
A14 12 A34 A46 2096 A61 A74 2 A93 A101 3 A121 49 A143 A152 1 A172 2 A191 A201 1
A11 42 A32 A42 7882 A61 A74 2 A93 A103 4 A122 45 A143 A153 1 A173 2 A191 A201 1
A11 24 A33 A40 4870 A61 A73 3 A93 A101 4 A124 53 A143 A153 2 A173 2 A191 A201 2
\end{lstlisting}
\vspace{0.5cm}

A preparação dos dados para a importação no WEKA consistiu na adição de uma seção de cabeçalho e da formatação dos dados para valores separados por vírgula (seção \ref{sec:prop_arff}), e o resultado é mostrado nas listagens \ref{lst:dev_arff_ger} e \ref{lst:dev_arff_aust} (essas listagens mostram apenas as primeiras cinco instâncias na seção de dados).

\vspace{0.5cm}
\begin{lstlisting}[caption=Arquivo ARFF do \emph{Cr.Ger}, label=lst:dev_arff_ger]
@relation cr.ger

@attribute A1  {A11,A12,A13,A14}
@attribute A2  numeric
@attribute A3  {A30,A31,A32,A33,A34}
@attribute A4  {A40,A41,A42,A43,A44,A45,A46,A47,A48,A49,A410}
@attribute A5  numeric
@attribute A6  {A61,A62,A63,A64,A65}
@attribute A7  {A71,A72,A73,A74,A75}
@attribute A8  numeric
@attribute A9  {A91,A92,A93,A94,A95}
@attribute A10 {A101,A102,A103}
@attribute A11 numeric
@attribute A12 {A121,A122,A123,A124}
@attribute A13 numeric
@attribute A14 {A141,A142,A143}
@attribute A15 {A151,A152,A153}
@attribute A16 numeric
@attribute A17 {A171,A172,A173,A174}
@attribute A18 numeric
@attribute A19 {A191,A192}
@attribute A20 {A201,A202}
@attribute A21 {1,2}

@data
A11,6,A34,A43,1169,A65,A75,4,A93,A101,4,A121,67,A143,A152,2,A173,1,A192,A201,1
A12,48,A32,A43,5951,A61,A73,2,A92,A101,2,A121,22,A143,A152,1,A173,1,A191,A201,2
A14,12,A34,A46,2096,A61,A74,2,A93,A101,3,A121,49,A143,A152,1,A172,2,A191,A201,1
A11,42,A32,A42,7882,A61,A74,2,A93,A103,4,A122,45,A143,A153,1,A173,2,A191,A201,1
A11,24,A33,A40,4870,A61,A73,3,A93,A101,4,A124,53,A143,A153,2,A173,2,A191,A201,2
\end{lstlisting}
\vspace{0.5cm}

\vspace{0.5cm}
\begin{lstlisting}[caption=Arquivo ARFF do \emph{Cr.Aust}, label=lst:dev_arff_aust]
@relation cr.aust

@attribute A1  {0,1}
@attribute A2  numeric
@attribute A3  numeric
@attribute A4  {1,2,3}
@attribute A5  {1,2,3,4,5,6,7,8,9,10,11,12,13,14}
@attribute A6  {1,2,3,4,5,6,7,8,9}
@attribute A7  numeric
@attribute A8  {1,0}
@attribute A9  {1,0}
@attribute A10 numeric
@attribute A11 {1,0}
@attribute A12 {1,2,3}
@attribute A13 numeric
@attribute A14 numeric
@attribute A15 {0,1}

@data
1,22.08,11.46,2,4,4,1.585,0,0,0,1,2,100,1213,0
0,22.67,7,2,8,4,0.165,0,0,0,0,2,160,1,0
0,29.58,1.75,1,4,4,1.25,0,0,0,1,2,280,1,0
0,21.67,11.5,1,5,3,0,1,1,11,1,2,0,1,1
1,20.17,8.17,2,6,4,1.96,1,1,14,0,2,60,159,1
\end{lstlisting}
\vspace{0.5cm}

\subsection{Interfaces}

O WEKA apresenta duas interfaces principais: linha de comando e gráfica (Interface Gráfica, GUI\nomenclature{GUI}{Graphical User Interface}). A interface gráfica é mais apropriada para exploração e experimentação, e para a apresentação dos dados, algoritmos e resultados. Porém, para usos mais avançados e experimentos mais complexos, a linha de comando é mais apropriada. Além da possibilidade muito maior de automação de tarefas e exposição de opções avançadas que não estão disponíveis na interface gráfica, essa interface consume menos recursos.

Nesse trabalho foi utilizada a interface por linha de comando, já que a repetição dos testes para cada algoritmo é muito mais fácil do que se utiliza-se a interface gráfica. Os exemplos de execução são apresentados conforme devem ser digitados em uma interface de linha de comando, em um emulador de terminal.

\subsection{Out of Memory}

Todas as aplicações java possuem um limite máximo de memória que pode ser alocado por cada processo da máquina virtual java (Java Virtual Machine, JVM\nomenclature{JVM}{Java Virtual Machine}). Caso não seja especificado, o valor padrão de 64\emph{mb} é utilizado. Para a maioria das aplicações, esse valor é insuficiente. É possível alterar esse valor através da opção \emph{-Xmx} para o executável, passando como parâmetro o novo valor para o limite. Para utilizar 4\emph{gb} como limite, por exemplo, pode-se utilizar as seguintes notações, onde \emph{k} e \emph{m} significam \emph{kilo} e \emph{mega}, respectivamente:

\vspace{0.5cm}
\begin{lstlisting}[caption=Opções para aumentar o limite de meória da JVM, label=lst:dev_java_xmx]
java -Xmx4294967296
java -Xmx4194304k
java -Xmx4096m
\end{lstlisting}
\vspace{0.25cm}
\centerline{Fonte: \cite{Oracle1995}}
\vspace{0.5cm}

\subsection{Filtros}

No WEKA, um filtro é um objeto que recebe um conjunto de dados como entrada e produz um conjunto de dados modificado. Esse é um processo comum da Mineração de Dados, chamado de pré-processamento dos dados: adicionar, remover ou alterar atributos, etc.

Um filtro comum, que é utilizado nesse trabalho, é o de criação de partições para o \emph{cross-validation}. Para esse filtro, são passados três argumentos. O argumento \emph{c} indica qual dos atributos é o atributo correspondente à classe, e é representado por um índice, iniciado em 1, conforme a declaração na seção de atributos do arquivo de dados (caso o padrão do WEKA seja usado, ou seja, o atributo de classe seja o último da listagem, pode ser utilizado o valor ``\emph{last}'' como argumento). O argumento \emph{N} indica o número de partições, e o argumento \emph{F} indica a partição selecionada.

Além desses, o argumento \emph{V} pode ser utilizado para gerar o conjunto inverso de seleções, útil para dividir o conjunto em duas partes complementares. Dessa forma, para gerar um conjunto de dados para testes e outro para treinamento, podem ser usados os seguintes comandos\footnote{Nesses exemplos, é usado o redirecionamento de entrada e saída presentes na maioria dos \emph{shells} UNIX. O caractere ``<'' seguido de um nome de arquivo indica que aquele arquivo será usado como entrada para o comando. De maneira semelhante, o caractere ``>'' seguido de um nome de arquivo indica que ele será usado como saída. O WEKA também permite que sejam utilizadas as opções \emph{i} e \emph{o}, respectivamente, para obter os mesmos resultados. No primeiro exemplo, a forma equivalente seria ``\emph{-i dataset.arff -o dataset\_test.arff}''.}:

\begin{lstlisting}[caption=Filtro para geração de partições para \emph{cross-validation}, label=lst:dev_filter]
java weka.filters.supervised.instance.StratifiedRemoveFolds -c last -N 4 -F 1 \
    < dataset.arff > dataset_test.arff
java weka.filters.supervised.instance.StratifiedRemoveFolds -c last -N 4 -F 1 -V \
    < dataset.arff > dataset_train.arff
\end{lstlisting}

\subsection{Execução de um teste}
