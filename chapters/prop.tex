\chapter{Proposta}

\section{Objetivos}

Conforme apresentado nos capítulos anteriores, a modelagem de um sistema de detecção de fraude com a utilização de técnicas inspiradas no sistema imunológico natural oferece diversas vantagens. Esse trabalho busca identificar exatamente \emph{quanto} esses modelos podem aperfeiçoar os métodos existentes. O método utilizado é a comparação dos resultados da classificação de algoritmos algoritmos tradicionais e Sistemas Imunológicos Artificiais sobre uma mesma base de dados.

\subsection{Bases de dados}

Para os testes, serão utilizadas duas bases contendo informações de contas de cartões de crédito. Essas bases de dados estão disponíveis publicamente, e fazem parte de um projeto chamado StatLog. Esse projeto foi concebido para testar diversos métodos de classificação em problemas grandes e comercialmente relevantes, comparando os resultados e determinando o quanto essas diferentes técnicas atendiam as necessidades da indústria. Conforme a própria descrição do projeto, os objetivos do projeto eram três:

\begin{enumerate}[a)]
    \item Possibilitar medidas críticas de desempenho para procedimentos de classificação disponíveis
    \item Indicar a natureza e escopo dos desenvolvimentos futuros necessários para que os métodos atendam as necessidades e expectativas dos usuários
    \item Indicar as direções mais promissoras de desenvolvimento para abordagens comercialmente imaturas.
\end{enumerate}

Os atributos do primeiro \emph{dataset} são descritos na listagem \ref{lst:german_dataset} \footnote{Na descrição dos atributos, DM significa \emph{deutsche mark} (marco alemão), a moeda corrente na Alemanha na época da coleta dos dados. Para efeito de comparação, o Banco Central Europeu estipulou a conversão irrevogável do marco alemão, a partir de 1º de janeiro de 1999, em DM 1.95583 = \euro 1 (http://www.ecb.int/press/pr/date/1998/html/pr981231\textunderscore{}2.en.html).}.

\begin{lstlisting}[caption=Atributos do primeiro dataset,label=lst:german_dataset]
    Atributo 1: (qualitativo)
    Situação da conta corrente existente
    A11 : ... < 0 DM
    A12 : 0 <= ... < 200 DM
    A13 : ... >= 200 DM
    A14 : sem conta corrente

    Atributo 2: (numérico)
    Duração em meses

    Atributo 3: (qualitativo)
    Histórico de crédito
    A30 : nenhum crédito retirado / todos os créditos pagos apropriadamente
    A31 : todos os créditos nesse banco pagos apropriadamente
    A32 : créditos existentes pagos apropriadamente até agora
    A33 : atraso no pagamento no passado
    A34 : conta crítica / outros créditos existentes (não nesse banco)

    Atributo 4: (qualitativo)
    Propósito
    A40 : carro (novo)
    A41 : carro (usado)
    A42 : móveis/equipamento
    A43 : rádio/televisão
    A44 : eletrodomésico
    A45 : reparos
    A46 : educação
    A47 : (férias - não existe no dataset)
    A48 : reciclagem profissional
    A49 : negócios
    A410 : outros

    Atributo 5: (numérico)
    Quantidade de crédito

    Atributo 6: (qualitativo)
    Poupança
    A61 : ... < 100 DM
    A62 : 100 <= ... < 500 DM
    A63 : 500 <= ... < 1000 DM
    A64 : .. >= 1000 DM
    A65 : desconhecido / sem poupança

    Atributo 7: (qualitativo)
    Emprego atual desde
    A71 : desempregado
    A72 : ... < 1 ano
    A73 : 1 <= ... < 4 anos
    A74 : 4 <= ... < 7 anos
    A75 : .. >= 7 anos

    Atributo 8: (numérico)
    Taxa de parcelamento em porcentagem do rendimento disponível

    Atributo 9: (qualitativo)
    Estado civil e sexo
    A91 : masculino : divorciado/separado
    A92 : feminino : divorciada/separada/casada
    A93 : masculino : solteiro
    A94 : masculino : casado/viúvo
    A95 : feminino : solteira

    Atributo 10: (qualitativo)
    Outros devedores / fiadores
    A101 : nenhum
    A102 : devedor solidário
    A103 : fiador

    Atributo 11: (numérico)
    Residência atual desde

    Atributo 12: (qualitativo)
    Propriedade
    A121 : imóvel
    A122 : se não A121 : financiamento / seguro de vida
    A123 : se não A121/A122 : carro ou outro, não incluso no atributo 6
    A124 : desconhecido / sem propriedade

    Atributo 13: (numérico)
    Idade em anos

    Atributo 14: (qualitativo)
    Outros planos de parcelamento
    A141 : banco
    A142 : lojas
    A143 : nenhum

    Atributo 15: (qualitativo)
    Residência
    A151 : alugada
    A152 : própria
    A153 : gratuita

    Atributo 16: (numérico)
    Número de créditos existentes nesse banco

    Atributo 17: (qualitativo)
    Emprego
    A171 : desempregado / sem proficiência / não-doméstico
    A172 : sem proficiência / doméstico
    A173 : proficiente / funcionário público
    A174 : administrador / auto-empregado / empregado altamente qualificado / oficial

    Atributo 18: (numérico)
    Número de dependentes

    Atributo 19: (qualitativo)
    Telefone
    A191 : nenhum
    A192 : sim, registrado sob o nom do consumidor

    Atributo 20: (qualitativo)
    Trabalhador estrangeiro
    A201 : sim
    A202 : não
\end{lstlisting}

\section{Arquitetura}

\section{Critérios}

Uma consideração importante quando analisa-se o desempenho de dois ou mais sistemas é que muito difícil (ou até mesmo impossível) excluir-se todos os fatores externos, resultando em uma análise completamente imparcial. Nesse caso específico, o resultado dos testes em cada algoritmo é fortemente influenciado pelas características dos dados na base de dados, embora os métodos de análise dos resultados procuram minimizar essa influência (conforme apresentado na seção \ref{sec:fraud_criteria}).

\section{Resultados}

\chapter{Conclusão}

\section{Síntese}

\section{TCC II}
