\chapter{Proposta}

Conforme apresentado nos capítulos anteriores, a modelagem de um sistema de detecção de fraude com a utilização de técnicas inspiradas no sistema imunológico natural busca trazer diversas vantagens tanto no processo de planejamento quanto na sua arquitetura final. Esse trabalho busca identificar exatamente \emph{quanto} esses modelos podem aperfeiçoar os métodos existentes. O método utilizado é a comparação dos resultados da classificação de algoritmos tradicionais e Sistemas Imunológicos Artificiais sobre um mesmo conjunto de dados. Esta seção apresenta os elementos principais para o desenvolvimento da proposta.

\iffalse

\section{Mineração de dados}

Introdução à mineração.
Etapas (figura).
Dados.
Pré-processamento.

\section{Critérios}

Critérios que serão utilizados para a comparação e justificativa.

\section{WEKA}

ALgoritmos inspirados no sistema imunológico que serão utilizados.

\subsection{CLONALG}
\subsection{AIRS}
\subsection{Immunos}
\fi

\section{Descrição geral}

Os principais elementos no desenvolvimento do trabalho são: os conjuntos de dados utilizados na execução dos testes, os algoritmos que serão testados nesses dados e os métodos de avaliação dos resultados.

\subsection{Descrição dos dados}

Para os testes serão utilizados dois conjuntos de dados contendo informações de contas de cartões de crédito. Esses dados estão disponíveis publicamente e fazem parte de um projeto chamado StatLog \cite{Michie1994}. Esse projeto foi concebido para testar diversos métodos de classificação em problemas grandes e comercialmente relevantes, comparando os seus resultados e determinando o quanto eles atendiam as necessidades da indústria. Conforme a sua própria descrição, os objetivos do projeto eram três:

\begin{enumerate}[a)]
    \item Possibilitar medidas críticas de desempenho para procedimentos de classificação disponíveis,
    \item Indicar a natureza e escopo dos desenvolvimentos futuros necessários para que os métodos atendam as necessidades e expectativas dos usuários e
    \item Indicar as direções mais promissoras de desenvolvimento para abordagens comercialmente imaturas.
\end{enumerate}

Os conjuntos de dados são denominados German Credit (Cr.Ger) e Australian Credit (Cr.Aust). Ambos contém informações sobre contas de crédito, conforme apresentado nas próximas seções.

\subsubsection{Cr.Ger}

Esse conjunto de dados foi cedido pelo professor Hans Hoffman, da Universidade de Hamburgo. Os atributos e valores que esses atributos podem assumir são descritos na listagem \ref{lst:ge_dataset} (traduzido do original)\footnote{Na descrição dos atributos, DM significa \emph{deutsche mark} (marco alemão), a moeda corrente na Alemanha na época da coleta dos dados. Para efeito de comparação, o Banco Central Europeu estipulou a conversão irrevogável do marco alemão, a partir de 1º de janeiro de 1999, em DM 1.95583 = \euro 1 (http://www.ecb.int/press/pr/date/1998/html/pr981231$\textunderscore$2.en.html).}.

\vspace{1cm}
\begin{lstlisting}[caption=Atributos do conjunto de dados alemão,label=lst:ge_dataset]
    Atributo 1: (qualitativo)
    Situação da conta corrente existente
    A11 : ... < 0 DM
    A12 : 0 <= ... < 200 DM
    A13 : ... >= 200 DM
    A14 : sem conta corrente

    Atributo 2: (numérico)
    Duração em meses

    Atributo 3: (qualitativo)
    Histórico de crédito
    A30 : nenhum crédito retirado / todos os créditos pagos apropriadamente
    A31 : todos os créditos nesse banco pagos apropriadamente
    A32 : créditos existentes pagos apropriadamente até agora
    A33 : atraso no pagamento no passado
    A34 : conta crítica / outros créditos existentes (não nesse banco)

    Atributo 4: (qualitativo)
    Propósito
    A40 : carro (novo)
    A41 : carro (usado)
    A42 : móveis/equipamento
    A43 : rádio/televisão
    A44 : eletrodomésico
    A45 : reparos
    A46 : educação
    A47 : (férias - não existe no conjunto de dados)
    A48 : reciclagem profissional
    A49 : negócios
    A410 : outros

    Atributo 5: (numérico)
    Quantidade de crédito

    Atributo 6: (qualitativo)
    Poupança
    A61 : ... < 100 DM
    A62 : 100 <= ... < 500 DM
    A63 : 500 <= ... < 1000 DM
    A64 : .. >= 1000 DM
    A65 : desconhecido / sem poupança

    Atributo 7: (qualitativo)
    Emprego atual desde
    A71 : desempregado
    A72 : ... < 1 ano
    A73 : 1 <= ... < 4 anos
    A74 : 4 <= ... < 7 anos
    A75 : .. >= 7 anos

    Atributo 8: (numérico)
    Taxa de parcelamento em porcentagem do rendimento disponível

    Atributo 9: (qualitativo)
    Estado civil e sexo
    A91 : masculino : divorciado/separado
    A92 : feminino : divorciada/separada/casada
    A93 : masculino : solteiro
    A94 : masculino : casado/viúvo
    A95 : feminino : solteira

    Atributo 10: (qualitativo)
    Outros devedores / fiadores
    A101 : nenhum
    A102 : devedor solidário
    A103 : fiador

    Atributo 11: (numérico)
    Residência atual desde

    Atributo 12: (qualitativo)
    Propriedade
    A121 : imóvel
    A122 : se não A121 : financiamento / seguro de vida
    A123 : se não A121/A122 : carro ou outro, não incluso no atributo 6
    A124 : desconhecido / sem propriedade

    Atributo 13: (numérico)
    Idade em anos

    Atributo 14: (qualitativo)
    Outros planos de parcelamento
    A141 : banco
    A142 : lojas
    A143 : nenhum

    Atributo 15: (qualitativo)
    Residência
    A151 : alugada
    A152 : própria
    A153 : gratuita

    Atributo 16: (numérico)
    Número de créditos existentes nesse banco

    Atributo 17: (qualitativo)
    Emprego
    A171 : desempregado / sem proficiência / não-doméstico
    A172 : sem proficiência / doméstico
    A173 : proficiente / funcionário público
    A174 : administrador / auto-empregado /
           empregado altamente qualificado / oficial

    Atributo 18: (numérico)
    Número de dependentes

    Atributo 19: (qualitativo)
    Telefone
    A191 : nenhum
    A192 : sim, registrado sob o nom do consumidor

    Atributo 20: (qualitativo)
    Trabalhador estrangeiro
    A201 : sim
    A202 : não
\end{lstlisting}

\subsubsection{Cr.Aust}

Os atributos do conjunto de dados australiano são descritos na listagem \ref{lst:prop_au_dataset} (traduzido do original). Uma grande desvantagem, muito comum nesse tipo de conjunto de dados (seção \ref{fraud:data}), é a de o nome dos campos ter sido alterado, perdendo o significado original. Para garantir a privacidade das informações contidas no conjunto de dados, provavelmente por questões competitivas, a empresa que cedeu os dados utilizou um processo de anonimização, garantindo que as informações confidenciais não possam ser recuperadas.

Os dados ainda podem ser utilizados para a validação de sistemas de detecção, mas não é possível saber de que forma o sistema chegou ao diagnóstico, tornando o resultado bem menos útil.

\vspace{1cm}
\begin{lstlisting}[caption=Atributos do conjunto de dados Cr.Aust, label=lst:prop_au_dataset]
A1: b, a
A2: contínuo
A3: contínuo
A4: u, y, l, t
A5: g, p, gg
A6: c, d, cc, i, j, k, m, r, q, w, x, e, aa, ff
A7: v, h, bb, j, n, z, dd, ff, o
A8: contínuo
A9: t, f
A10: t, f
A11: contínuo
A12: t, f
A13: g, p, s
A14: contínuo
A15: contínuo
A16: +,- (atributo de classe)
\end{lstlisting}

\subsection{Algoritmos}

Os algoritmos que serão utilizados para a comparação são um pacote de algoritmos desenvolvido por Jason Brownlee \cite{Brownlee2011}, na versão mais atual (1.8, maio de 2011). Esse pacote foi especialmente desenvolvido para a plataforma de aprendizagem de máquina Weka (descrita na seção \ref{sec:prop_weka}) e são disponibilizados através de uma licença aberta (GNU GPL). Nele, são implementados diversas categorias de algoritmos de Redes Neurais e Sistemas Imunológicos Artificiais:

\begin{enumerate}
    \item Redes neurais
    \begin{enumerate}[a)]
        \item \textbf{Learning Vector Quantization (LVQ)}: similar às redes neurais, que utiliza aprendizagem supervisionada, baseada em protótipos, para classificação de dados. É similar ao algoritmo de k-vizinhos mais próximos e um precursor do próximo algoritmo.
        \item \textbf{Self-Organizing Map (SOM)}: algoritmo de redes neurais que utiliza aprendizagem não-supervisionada. Mapeia os valores da base de dados produzindo um mapa: um espaço de duas dimensões, uma representação discreta dos dados de entrada.
        \item \textbf{Feed-Forward Artificial Neural Network (FF-ANN)}: tipo de algoritmo de redes neurais onde o sinal é propagado na direção entrada-saída, sem conexões de \emph{feedback}.
    \end{enumerate}
    \item Sistemas Imunológicos Artificiais
    \begin{enumerate}[a)]
        \item \textbf{Artificial Immune Recognition System (AIRS)}: algoritmo imunológico supervisionado descrito na seção \ref{sec:prop_airs}.
        \item \textbf{Clonal Selection Algorithm (CLONALG)}: um dos principais algoritmos imunológicos, o algoritmo da seleção clonal, descrito na seção \ref{sec:ais_clonalg}.
        \item \textbf{Immunos-81}: algoritmo imunológico descrito na seção \ref{sec:prop_immunos}.
    \end{enumerate}
\end{enumerate}

Desses, os algoritmos de Sistemas Imunológicos Artificiais (2) serão utilizados para comparação com os métodos tradicionais. O algoritmo da seleção clonal (CLONALG) foi descrito na seção \ref{sec:ais_clonalg}. Os outros dois são descritos nas próximas seções. A figura \ref{fig:prop_wekaais} mostra esses algoritmos conforme apresentados na interface do WEKA.

\begin{figure}[h!]
    \centering
    \caption{Algoritmos no WEKA}
    \label{fig:prop_wekaais}
    \includegraphics[width=0.75\textwidth]{img/weka_ais.png}
\end{figure}

\subsubsection{AIRS}
\label{sec:prop_airs}

Esse algoritmo foi criado por Andrew Watkins \cite{Andrew2003} e tinha como objetivo implementar um Sistema Imunológico Artificial que utilizasse aprendizagem supervisionada. Na época de seu desenvolvimento, a área de algoritmos imunológicos supervisionados não havia sido explorada, apesar da pesquisa intensa na área de algoritmos imunológicos não-supervisionados \footnote{De acordo com o autor, o único outro algoritmo imunológico supervisionado era o Immunos, apresentado na próxima seção.}.

Por esse motivo, esse foi o primeiro algoritmo supervisionado a implementar a maior parte das técnicas inspiradas nos sistemas imunológicos, como: modelagem dos dados como antígenos e anticorpos, expansão clonal dos linfócitos, mutação e maturação de afinidade e memória imunológica.

Na definição desse algoritmo também foi aplicado o formalismo do espaço de formas. Conforme apresentado na seção \ref{sec:ais_shape}, esse é um formalismo bastante apropriado para a modelagem de Sistemas Imunológicos Artificiais, devido a forte semelhança entre o espaço de formas e as diferentes formas que os receptores dos linfócitos no sistema imunológico apresentam. Os antígenos identificados por esses receptores formam a região de reconhecimento ao redor de cada ponto no espaço de formas. Em um algoritmo de aprendizagem, isso representa a correspondência entre as instâncias de treinamento (antígenos) e as possíveis soluções (células B).

\subsubsection{Immunos}
\label{sec:prop_immunos}

Esse algoritmo foi apresentado por Jerome Carter \cite{Carter2000}. O algoritmo Imunnos foi desenvolvido em oposição aos algoritmos que tentavam simular fielmente o comportamento do sistema imunológico. Segundo o autor, do ponto de vista da computação, construir um sistema que utilize equações cuidadosamente derivadas dos estudos teóricos a imunologia seria um feito notável, mas não ideal. Os elementos do sistema imunológico foram reduzidos ao nível mais fundamental para que fossem introduzidos no sistema.

O principal elemento na modelagem do algoritmo foi a utilização de pequenas e simples unidades de processamento conectadas em paralelo, como pode ser observado nos nodos das redes neurais e linfócitos do sistema imunológico. As principais metas da modelagem eram:

\begin{enumerate}[a)]
    \vspace{2mm}
    \itemsep1pt
    \item Representação interna simples de ser entendida,
    \item Capacidade de generalização sobre os dados de entrada,
    \item Tempos de treinamento previsíveis,
    \item Aprendizagem \emph{online},
    \item Potencial para atuar como memória associativa,
    \item Suporte a atributos contínuos e qualitativos,
    \item Capacidade de aprendizagem e recuperação de um grande número de padrões,
    \item Aprendizagem baseada em experiência e
    \item Aprendizagem supervisionada.
    \vspace{2mm}
\end{enumerate}

\subsubsection{Algoritmos para comparação}

Para a comparação dos algoritmos apresentados na seção anterior, os resultados serão comparados com os resultados de algoritmos tradicionais da área da Inteligência Artificial.

\begin{enumerate}[a)]
    \item Redes neurais
    \item Máquina de vetor de suporte
    \item Algoritmos genéticos
    \item Árvores de decisão\iffalse (j48) (http://www.slideshare.net/butest/weka-tutorial) \fi
    \item Learning Vector Quantization
\end{enumerate}

Implementações de todos esses algoritmos estão disponíveis na versão oficial da ferramenta WEKA. Isso facilita a comparação dos resultados com os resultados dos algoritmos imunológicos.

\subsection{Weka}{}
\label{sec:prop_weka}

Como plataforma de testes será utilizado o \emph{software} Weka. Weka (\emph{Waikato Environment for Knowledge Analysis}, Ambiente para Aprendizagem de Máquina de Waikato) é uma suite de aplicações de aprendizagem de máquina desenvolvida na Universidade de Waikato na Nova Zelândia. Essa ferramenta é largamente utilizada em projetos nessa área, devido a sua licença aberta (GNU GPL), que permite que seja utilizada quase sem restrições.

A figura \ref{fig:prop_weka} mostra uma captura de tela do programa em execução, mostrando a visualização de um conjunto de dados de exemplo.

\begin{figure}[h!]
    \centering
    \caption{Janela do módulo Explorer - Weka 3.6.4}
    \label{fig:prop_weka}
    \includegraphics[width=0.75\textwidth]{img/weka.png}
\end{figure}

\iffalse arff file format \fi

A listagem \ref{lst:prop_weka_out} mostra um exemplo dos dados de saída após a execução de um teste de um dos algoritmos (LVQ) em um conjunto de dados de teste.

\vspace{1cm}
\begin{lstlisting}[caption=Exemplo de saída de uma execução do WEKA, label=lst:prop_weka_out]
    === Run information ===

    Scheme:       weka.classifiers.neural.lvq.MultipassLvq -A
    "weka.classifiers.neural.lvq.Olvq1 -M 1 -C 20 -I 1000 -L 1 -R 0.3 -S 1 -G
    false" -B "weka.classifiers.neural.lvq.Lvq3 -M 1 -C 20 -I 10000 -L 1 -R 0.05 -S
    1 -G false -W 0.3 -E 0.1"
    Relation:     weather
    Instances:    14
    Attributes:   5
                  outlook
                  temperature
                  humidity
                  windy
                  play
    Test mode:    10-fold cross-validation

    === Classifier model (full training set) ===

    -- Training Time Breakdown --
    Pass 0: 17ms
    Pass 1: 51ms
    Total Model Preparation Time: 68ms

    -- Cass Distribution --
    yes :  15 (75%)
    no :  5 (25%)



    Time taken to build model: 0.07 seconds

    === Stratified cross-validation ===
    === Summary ===

    Correctly Classified Instances           5               35.7143 %
    Incorrectly Classified Instances         9               64.2857 %
    Kappa statistic                         -0.4651
    Mean absolute error                      0.6429
    Root mean squared error                  0.8018
    Relative absolute error                135      %
    Root relative squared error            162.5137 %
    Total Number of Instances               14

    === Detailed Accuracy By Class ===

                   TP Rate   FP Rate   Precision   Recall  F-Measure   ROC Area
    Class
                     0.556     1          0.5       0.556     0.526      0.278
    yes
                     0         0.444      0         0         0          0.278
    no
    Weighted Avg.    0.357     0.802      0.321     0.357     0.338      0.278

    === Confusion Matrix ===

     a b   <-- classified as
     5 4 | a = yes
     5 0 | b = no
\end{lstlisting}

\section{Método de pesquisa}

As atividades da segunda etapa do Trabalho de Conclusão de Curso serão desenvolvidas conforme apresentadas na tabela \ref{tab:prop_cron} e serão descritas à seguir.

A etapa de coleta e preparação de dados envolverá a adaptação dos dados nos conjuntos ao formato de entrada esperado pelo WEKA, já que eles se encontram em estado bruto. O formato mais comum para utilização nesse programa são arquivos \emph{Attribute Relationship File Format} (ARFF, Formato de Arquivo de Atributo-Relação). Um arquivo nesse formato é apresentado na listagem \ref{lst:prop_arff}. Esse mesmo arquivo foi apresentado na figura \ref{fig:prop_weka}, mostrando a visualização pela interface do WEKA.

\vspace{1cm}
\begin{lstlisting}[caption=Exemplo de arquivo no formato ARFF, label=lst:prop_arff]
    @relation weather

    @attribute outlook {sunny, overcast, rainy}
    @attribute temperature real
    @attribute humidity real
    @attribute windy {TRUE, FALSE}
    @attribute play {yes, no}

    @data
    sunny,85,85,FALSE,no
    sunny,80,90,TRUE,no
    overcast,83,86,FALSE,yes
    rainy,70,96,FALSE,yes
    rainy,68,80,FALSE,yes
    rainy,65,70,TRUE,no
    overcast,64,65,TRUE,yes
    sunny,72,95,FALSE,no
    sunny,69,70,FALSE,yes
    rainy,75,80,FALSE,yes
    sunny,75,70,TRUE,yes
    overcast,72,90,TRUE,yes
    overcast,81,75,FALSE,yes
    rainy,71,91,TRUE,no
\end{lstlisting}
\vspace{1cm}

Ainda, podem ser necessários ajustes específicos nos dados para que possam servir como entrada para alguns dos algoritmos, caso estes não tenham suporte aos tipos de dados. Com os dados prontos, iniciará a fase de execução dos diferentes algoritmos sobre esses conjuntos de dados. A coleta dos resultados será feita através dos dados de saída, mostrados na listagem \ref{lst:prop_weka_out}.

Uma característica positiva da ferramenta WEKA é a possibilidade de alterar parâmetros tanto do algoritmo em si (caso eles ofereçam essa funcionalidade) quanto da execução do teste em si. Assim, pode-se executar diversos testes, variando esses parâmetros, para ao final comparar também essas configurações. A princípio, será definida apenas uma configuração, que será usada para a execução de todos os testes. No entanto, conforme for possível, poderão ser executados testes variando essa configuração, gerando comparação mais diversificadas.

De posse dos resultados dos testes de todos os algoritmos, iniciará a etapa de análise. Os resultados dos algoritmos imunológicos serão comparados com os resultados dos algoritmos tradicionais. Essa comparação envolve diversos níveis: taxas de erros e acertos (conforme apresentado na seção \ref{sec:fraud_criteria}), tempos de execução (que inclui tempo de treinamento do algoritmo e o tempo dos testes), utilização de recursos, etc. Para a visualização dos resultados, serão criados elementos visuais, como tabelas e gráficos.

Esse período também inclui a documentação do processo através da redação da monografia e a apresentação ao fim do semestre.

\begin{table}[h]
    \vspace{1cm}
    \caption{Cronograma}
    \centering
    \begin{tabular}{l >{\arraybackslash}m{4cm} >{\centering\arraybackslash}m{7cm} c}
        \multicolumn{2}{c}{Etapa} & Resultado & Data \\
        \hline
        1 & Coleta e preparação dos dados       & Arquivos no formato esperado pelo WEKA & março/2013 \\
        2 & Aplicação dos algoritmos            & Resultados da execução (taxas de acerto, métricas, tempo de execução, listagem \ref{lst:prop_weka_out} & abril/2013 \\
        3 & Comparação e análise dos resultados & Ranking de resultados & maio/2013 \\
        4 & Conclusão e contribuições           & Análise & junho/2013 \\
        5 & Redação e revisão                   & & março-julho/2013 \\
        6 & Apresentação                        & & julho/2013 \\
    \end{tabular}
    \label{tab:prop_cron}
    \vspace{1cm}
\end{table}
