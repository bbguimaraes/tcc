\chapter{Introdução}

O crescente avanço na tecnologia de armazenamento de dados, principalmente em termos de velocidade e tamanho, permite a criação de bancos de dados complexos e detalhados. Com isso, o desenvolvimento de programas de computador que há anos atrás seriam computacionalmente inviáveis torna-se possível. Uma área fortemente influenciada por esse fator é a mineração de dados.

A Mineração de Dados\nomenclature{MD}{Mineração de Dados} é associada à ideia do aproveitamento de grandes bases de dados para auxiliar o profissional em alguma tarefa, através da apresentação de padrões e relações que seriam difíceis ou impossíveis de serem encontrados pelo observador humano. Ela foi definida como ``análise de bases (geralmente extensas) de dados previamente coletados para estabelecer relações e apresentar a informação de forma mais clara e útil ao proprietário dos dados'' \cite[p. 1]{Hand2001}\footnote{``Data mining is the analysis of (often large) observational data sets to find unsuspected relationships and to summarize the data in novel ways that are both understandable and useful to the data owner.''.} As atividades envolvidas na mineração de dados são:

\begin{enumerate}[a)]
    \item \textbf{Determinar como os dados a serem usados serão representados}. As estruturas do problema devem ser codificadas para que possam ser utilizadas por algoritmos computacionais. Formas comuns de representação são cadeias de bits, números inteiros, números reais e valores categóricos.
    \item \textbf{Decidir a função de aptidão}. Essa função é utilizada para quantificar a adequação de um modelo a um conjunto de dados, permitindo definir se um modelo é "melhor" que outro, ou seja, representa mais precisamente os elementos daquele conjunto. Essa função é altamente dependente da definição da representação dos dados, e é uma das partes mais importantes da definição do sistema, já que é ela que controla a convergência para uma solução ótima.
    \item \textbf{Escolher um processo algorítmico para otimizar a função de comparação}. Através de um algoritmo específico, modelos diferentes (ou parâmetros diferentes para um mesmo tipo de modelo) são criados e testados utilizando a função de aptidão. Esse processo é repetido até que a condição de término seja encontrada, geralmente após um determinado limite da função de aptidão ou um determinado número de iterações.
    \item \textbf{Administrar o acesso aos dados de forma eficiente durante a execução dos algoritmos}. Grandes \emph{datasets} geralmente excedem o tamanho dos dispositivos de armazenamento primário atuais. Dispositivos de armazenamento secundário ou terciário são utilizados, tornando o acesso a esses dados ineficiente. Um algoritmo corre o risco de tornar-se computacionalmente inviável caso o programador não considere esse fato.
\end{enumerate}

Os exemplos da utilização da Mineração de Dados são inúmeros, estendendo-se virtualmente a qualquer atividade, de empresas comerciais à medicina e engenharia. A coleta e armazenamento de informações cresce constantemente, tornando a análise de dados impossível sem o uso de uma ferramenta automatizada. A mineração extrai padrões ou modelos de fontes de informação que seriam difíceis ou impossíveis de serem visualizadas apenas observando-se os dados. Isso tem motivado o desenvolvimento de algoritmos capazes de identificar padrões com mais detalhamento e significado.

Como área interdisciplinar, a mineração atrai estudiosos de diversas áreas da computação como, em especial, a Inteligência Artificial\nomenclature{IA}{Inteligência Artificial}. A combinação de técnicas tradicionais de mineração com os algoritmos da Inteligência Artificial estende em muito o seu poder de análise. Toma-se a definição básica de Luger \cite[p. 1]{Luger2009}:

\begin{quote}
``Inteligência Artificial (IA) pode ser definida como a área da ciência da computação que se dedica a automação de comportamento inteligente.''\footnote{``Artificial intelligence (AI) may be defined as the branch of computer science that is concerned with the automation of inteligent behaviour.''.}
\end{quote}

Analisar informações, raciocinar e tirar conclusões são também objeto de estudo da Inteligência Artificial. Nessa área, alguns dos métodos mais aplicados são redes neurais e bayesianas, \emph{clustering} e lógica nebulosa (\emph{fuzzy}). De fato, as grandes bases de dados, que geralmente são objeto de atuação da mineração, constituem excelente campo de testes para os algoritmos da IA. 

Uma área que têm recebido crescente atenção de estudos de mineração de dados e Inteligência Artificial é a de segurança da informação. O desenvolvimento de ferramentas de mineração é visto como um forte candidato à solução de problemas nessa área, já que elas são capazes de extrair modelos e comportamentos da base de dados que dificilmente seriam percebidos por um observador humano. Esses modelos podem ser usados para identificar padrões em novas instâncias, automatizando o processo de monitoramento. O ganho que um profissional tem ao utilizar uma ferramenta dessa natureza é enorme: a tarefa de análise de bases de dados extensas é repassada dele para o computador.

O reconhecimento de padrões e a capacidade de investigação de grandes bases de dados são os maiores desafios que profissionais da segurança da informação enfrentam. Nesse sentido, o objetivo é repassar à maquina cada vez mais o trabalho mecânico, para que o profissional possa se dedicar à análise crítica do material compilado. Ou ainda um algoritmo de mineração de dados pode analisar a grande quantidade de dados, deixando para os especialistas humanos a análise apenas daquelas identificadas como suspeitas.

Dentro da segurança da informação, escolheu-se um contexto mais específico para delimitar o escopo do trabalho: a detecção de fraude. Na legislação brasileira \cite[p. 324]{DePlacido1982}, a fraude é definida como ``o \emph{engano malicioso} ou a \emph{ação astuciosa}, promovidos de \emph{má fé}, para \emph{ocultação da verdade} ou \emph{fuga ao cumprimento do dever}''. Em seu artigo \emph{Measuring the impact of fraude in the UK}, \citet{Levi2008} definem a fraude como um mecanismo através do qual o fraudador ganha uma vantagem ilegal ou causa perdas ilegais. Os tipos mais proeminentes de fraude ocorrem na área médica, nos seguros (de vida, casa, automóveis, etc), cartões de crédito e telecomunicações.

A fraude não restringe-se aos casos em que há perdas financeiras diretas. Em um ambiente competitivo, a fraude pode ser um problema crítico se as medidas preventivas e planos de contingência não forem planejados e aplicados. Segundo a agência americana de investigação FBI \cite{FBI2010}, o custo das fraudes na área de seguros, excluindo-se os seguros de saúde, apenas nos Estados Unidos, é estimado em 40 bilhões de dólares por ano. O aumento médio nas apólices de uma família comum devido à fraude é calculado em 400 a 700 dólares. A \emph{National Health Care Anti-Fraud Association} estima que 3\% de todos os gastos médicos dos Estados Unidos são perdidos por causa de fraudes, o que representa uma quantia de 68 milhões de dólares anualmente.

A detecção de fraude é uma das etapas de um processo maior, comumente denominado de \emph{controle de fraude}. Ela consiste na identificação de padrões e comportamentos associados à atividades irregulares, prevenindo que elas se concretizem \cite{Phua2010}. A aplicação de técnicas de mineração de dados em aplicações desse tipo proporcionou espaço para o desenvolvimento de incontáveis trabalhos \cite{Phua2010}. Essa é uma das principais aplicações da mineração de dados em aplicações corporativas e governamentais.

Dados recentes mostram que a detecção de fraude tradicional, o processo manual conduzido por um especialista humano é caro, devido ao custo da contratação desse especialista e ao grande volume de dados presente nos bancos de dados. Outro problema associado a essa prática é a incapacidade de detecção de padrões de fraude dispersos entre os dados, além do trabalho exaustivo e gasto de recursos que poderiam ser aplicados em áreas mais produtivas. O tempo necessário para que essa informação seja reunida, compilada e analisada também torna-se um fator proibitivo, já que dá ao fraudador uma janela de tempo muito grande para agir e até mesmo ocultar seus feitos. Sistemas de detecção de fraude têm o objetivo de automatizar e reduzir as partes manuais desse processo de verificação.

Com frequência, nas ciências, pesquisadores se voltam a áreas diferentes das suas próprias, com o objetivo de encontrar soluções para os seus problemas adaptando outras soluções. Um exemplo disso, na Inteligência Artificial, são os Sistemas Imunológicos Artificiais, inspirados no sistema imunológico natural. Conforme definiu Castro:

\begin{quote}
``Sistemas Imunológicos Artificiais (SIA) são sistemas adaptativos, inspirados na teoria da imunidade e funções, princípios e modelos imunológicos observadas, que são aplicados à resolução de problemas.''\footnote{``Artificial Immune Systems (AIS) are adaptive systems, inspired by theoretical immunology and observed immune functions, principles and models, which are applied to problem solving.''. CASTRO, L. N.; TIMMIS, J. 2002. \emph{Artificial Immune Systems: A New Computational Intelligence Approach.} p. 57.}
\end{quote}

O sistema imunológico natural é um sistema robusto, complexo e adaptativo, que classifica as células do corpo como próprias e não-próprias. Isso é feito através de uma força de trabalho distribuída, capaz de agir de forma local ou global, usando uma rede de mensagens químicas para comunicar-se \cite{Aickelin2005}.

Modelos computacionais foram criados para tentar adaptar o funcionamento do sistema imunológico natural a problemas da computação, e dá-se a esses modelos o nome de Sistemas Imunológicos Artificiais. A utilização desse tipo de sistema é relativamente nova: os primeiros trabalhos nessa área foram desenvolvidos a partir da metade dos anos 90. Mesmo assim é uma técnica que apresenta bons resultados e tem despertado o interesse de desenvolvedores de sistemas de segurança da informação. A modelagem desse tipo de sistema como um Sistema Imunológico Artificial oferece diversas vantagens:

\begin{enumerate}[a)]
\item O sistema imunológico natural protege o corpo de invasões externas. Essa metáfora pode ser facilmente estendida para problemas de segurança, facilitando a modelagem dos componentes do sistema.
\item A natureza distribuída do sistema imunológico facilita a implementação do sistema seguindo o paradigma dos sistemas distribuídos da computação.
\item O sistema imunológico tem características que são muito importantes nos sistemas de segurança da informação. Ele contém uma memória, que reconhece agentes que o corpo já teve contato e apresenta um comportamento adaptativo, capaz de reconhecer e combater novas ameaças que ainda não haviam sido tratadas.
\end{enumerate}

\citet{Dasgupta2010} apresentam os trabalhos recentes na área de Sistemas Imunológicos Artificiais. A tabela \ref{ais:applications} mostra as áreas onde esses sistemas foram aplicados.

\begin{table}[h!]
    \vspace{1cm}
    \centering
    \caption{Aplicações recentes \cite{Dasgupta2010}}
    \begin{tabular}{l c r}
        \\
        \hline
        Mineração de dados                                    \\
        Redes e segurança                                     \\
        Otimização                                            \\
        Automação e design                                    \\
        Detecção de anomalia                                  \\
        Bioinformática                                        \\
        Processamento de texto                                \\
        Reconhecimento de padrões, clustering e classificação \\
        \hline
    \end{tabular}
    \label{ais:applications}
    \vspace{1cm}
\end{table}

Uma adição ainda mais recente foi a da Teoria do Perigo, proposta por Matzinger, em 1994. A principal revolução na Teoria do Perigo é a mudança na forma como as células tomam conhecimento e reagem, introduzindo a noção de perigo e tolerância. Essa mudança não é tanto na modelagem e representação dos dados, mas sim na decisão do que deve ser modelado, e o que deve gerar respostas do sistema \cite{Aickelin2005}. A adição da Teoria do Perigo a um sistema, além das vantagens proporcionadas pela modelagem como Sistema Imunológico Artificial, restringe o domínio do problema, identificando o subconjunto de atributos mais importantes à detecção, e facilita a implementação em ambientes onde a definição de perigo muda com facilidade, características importantes em sistemas de segurança da informação.

\iffalse

\section{Objetivos}
\section{Organização}

\fi
