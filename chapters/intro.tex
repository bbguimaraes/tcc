\chapter{Introdução}

O crescente avanço na tecnologia de armazenamento de dados, principalmente em termos de velocidade e tamanho, permite a criação de bancos de dados complexos e detalhados. Com isso, o desenvolvimento de programas de computador que há anos atrás seriam computacionalmente inviáveis torna-se possível. Uma área fortemente influenciada por esse fator é a mineração de dados. Definida como ``análise de bases (geralmente extensas) de dados previamente coletados para estabelecer relações e apresentar a informação de forma mais clara e útil ao proprietário dos dados''\footnote{``Data mining is the analysis of (often large) observational data sets to find unsuspected relationships and to summarize the data in novel ways that are both understandable and useful to the data owner.''. HAND, D. J.; MANILLA, H.; SMYTH, P. 2001. \emph{Principles of data mining}. p. 1.}, a mineração de dados é associada à ideia do aproveitamento de grandes bases de dados para auxiliar o profissional em alguma tarefa, através da apresentação de padrões e relações que seriam difíceis ou impossíveis de serem encontrados pelo observador humano.

Os exemplos da sua utilização são inúmeros, estendendo-se virtualmente a qualquer atividade, de empresas comerciais à medicina e engenharia. Cada vez mais informação coletada e armazenada, e a análise torna-se impossível sem o uso de uma ferramenta automatizada para sintetizá-la. A mineração extrai padrões ou modelos dessas fontes de informação, apresentando informação que pode ser utilizada gerando alguma vantagem para o detentor dos dados. Isso tem motivado pesquisadores a desenvolver algoritmos capazes de identificar esses padrões com mais detalhamento e significado.

Como área interdisciplinar, a mineração atrai estudiosos de diversas áreas da computação como, em especial, a Inteligência Artificial (IA). A combinação de técnicas tradicionais de mineração com os algoritmos desta área estende em muito o seu poder de análise e sintetização. Até a definição mais básica da Inteligência Artificial já mostra que as duas áreas têm objetivos em comum, como a ``tentativa de definição'' de Luger:

\begin{quote}
``Inteligência Artificial (IA) pode ser definida como a área da ciência da computação que se dedica a automação de comportamento inteligente.''\footnote{``Artificial intelligence (AI) may be defined as the branch of computer science that is concerned with the automation of inteligent behaviour.''. LUGER, G. F. 2008. \emph{Artificial Intelligence: Structures and Strategies for Complex Problem Solving}. p. 1.}
\end{quote}

Analisar informações, raciocinar e tirar conclusões são também objeto de estudo da Inteligência Artificial, e muitos dos seus métodos podem ser usados em conjunto com a mineração. De fato, as grandes bases de dados que geralmente são objeto de atuação da mineração constituem excelente campo de testes para os algoritmos da IA. Redes neurais e bayesianas, clustering e lógica nebulosa (fuzzy) são alguns dos métodos mais aplicados nessas situações.

Com frequência, nas ciências, pesquisadores se voltam a áreas diferentes das suas próprias, com o objetivo de encontrar soluções para os seus problemas adaptando outras soluções. Um exemplo disso, na Inteligência Artificial, são os Sistemas Imunológicos Artificiais, inspirados no sistema imunológico natural. Conforme definiu Castro:

\begin{quote}
``\emph{Sistemas Imunológicos Artificiais (SIA) são sistemas adaptativos, inspirados na teoria da imunidade e funções, princípios e modelos imunológicos observadas, que são aplicados à resolução de problemas.''\footnote{``Artificial Immune Systems (AIS) are adaptive systems, inspired by theoretical immunology and observed immune functions, principles and models, which are applied to problem solving.''. CASTRO, L. N.; TIMMIS, J. 2002. \emph{Artificial Immune Systems: A New Computational Intelligence Approach.} p. 57.}}
\end{quote}

A utilização desse tipo de sistema é nova, até dentro da área de Inteligência Artificial: os primeiros trabalhos nessa área foram desenvolvidos a partir da metade dos anos 90. Mesmo assim, é uma técnica que apresenta bons resultados e tem despertado o interesse de pesquisadores. Uma adição ainda mais recente foi a da Teoria do Perigo, proposta por Matzinger, em 1994. A principal revolução na Teoria do Perigo é a mudança na forma como as células tomam conhecimento e reagem, introduzindo a noção de perigo e tolerância.

Uma área que têm recebido crescente atenção de estudos de mineração de dados e Inteligência Artificial é a de segurança. O reconhecimento de padrões e capacidade de investigação de grandes bases de dados são os maiores desafios que profissionais da segurança enfrentam, não apenas na área da informática, mas de forma geral. O desenvolvimento de ferramentas de mineração é visto como um forte candidato à solução desses problemas. O ganho que um profissional tem ao utilizar uma ferramenta dessa natureza é enorme: a tarefa de análise de bases de dados extensas é passada dele para o computador. Nesse sentido, o objetivo é relevar à maquina cada vez mais o trabalho mecânico, para que o ser humano possa se dedicar à análise crítica do material compilado.

Dentro da segurança, escolheu-se uma área mais específica para delimitar o escopo do trabalho. A detecção de fraude consiste na identificação de padrões e comportamentos associados à atividades irregulares, prevenindo que elas se concretizem. A aplicação de técnicas de mineração de dados em aplicações desse tipo proporcionou espaço para o desenvolvimento de incontáveis trabalhos4. No entanto, a utilização de Sistemas Imunológicos Artificiais ainda é pouco explorada. Viu-se nesse fato uma oportunidade de exploração nesse trabalho.

\section{Objetivos}
\section{Organização}

