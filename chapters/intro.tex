\chapter{Introdução}

O crescente avanço na tecnologia de armazenamento de dados, principalmente em termos de velocidade e tamanho, permite a criação de bancos de dados complexos e detalhados. Com isso, o desenvolvimento de programas de computador que há anos atrás seriam computacionalmente inviáveis torna-se possível. Uma área fortemente influenciada por esse fator é a mineração de dados.

Definida como ``análise de bases (geralmente extensas) de dados previamente coletados para estabelecer relações e apresentar a informação de forma mais clara e útil ao proprietário dos dados'' \cite[p. 1]{Hand2001}\footnote{``Data mining is the analysis of (often large) observational data sets to find unsuspected relationships and to summarize the data in novel ways that are both understandable and useful to the data owner.''.}, a mineração de dados é associada à ideia do aproveitamento de grandes bases de dados para auxiliar o profissional em alguma tarefa, através da apresentação de padrões e relações que seriam difíceis ou impossíveis de serem encontrados pelo observador humano. As atividades envolvidas na mineração de dados são:

\begin{itemize}
\item Determinar como os dados a serem usados serão representados. As estruturas do problema devem ser codificadas para que possam ser utilizadas por algoritmos computacionais. Formas comuns de representação são cadeias de bits, números inteiros, números reais e valores categóricos.
\item Decidir a função de aptidão. Essa função é utilizada para quantificar a adequação de um modelo a um conjunto de dados, permitindo definir se um modelo é "melhor" que outro, ou seja, representa mais precisamente os elementos daquele conjunto.
\item Escolher um processo algorítmico para otimizar a função de comparação. Através de um algoritmo específico, modelos diferentes (ou parâmetros diferentes para um mesmo tipo de modelo) são criados e testados utilizando a função de aptidão. Esse processo é repetido até que a condição de término seja encontrada, geralmente após um determinado limite da função de aptidão ou um determinado número de iterações.
\item Administrar o acesso aos dados de forma eficiente durante a execução dos algoritmos. Grandes \emph{datasets} geralmente excedem o tamanho dos dispositivos de armazenamento primário atuais. Dispositivos de armazenamento secundário ou terciário são utlizados, tornando o acesso a esses dados ineficiente. Um algoritmo corre o risco de tornar-se computacionalmente inviável caso o programador não considere esse fato.
\end{itemize}

Os exemplos da sua utilização são inúmeros, estendendo-se virtualmente a qualquer atividade, de empresas comerciais à medicina e engenharia. A coleta e armazenamento de informações cresce constantemente, tornando a análise desses dados impossível sem o uso de uma ferramenta automatizada. A mineração extrai padrões ou modelos dessas fontes de informação, que seriam difíceis ou impossíveis de serem visualizadas apenas observando-se os dados. Isso tem motivado o desenvolvimento de algoritmos capazes de identificar padrões com mais detalhamento e significado.

Como área interdisciplinar, a mineração atrai estudiosos de diversas áreas da computação como, em especial, a Inteligência Artificial. A combinação de técnicas tradicionais de mineração com os algoritmos da Inteligência Artificial estende em muito o seu poder de análise. Tomando a definição básica de Luger \cite[p. 1]{Luger2009}:

\begin{quote}
``Inteligência Artificial (IA) pode ser definida como a área da ciência da computação que se dedica a automação de comportamento inteligente.''\footnote{``Artificial intelligence (AI) may be defined as the branch of computer science that is concerned with the automation of inteligent behaviour.''.}
\end{quote}

Analisar informações, raciocinar e tirar conclusões são também objeto de estudo da Inteligência Artificial. Redes neurais e bayesianas, clustering e lógica nebulosa (\emph{fuzzy}) são alguns dos métodos mais aplicados nessas situações. De fato, as grandes bases de dados que geralmente são objeto de atuação da mineração constituem excelente campo de testes para os algoritmos da IA. 

Uma área que têm recebido crescente atenção de estudos de mineração de dados e Inteligência Artificial é a de segurança da informação. O desenvolvimento de ferramentas de mineração é visto como um forte candidato à solução desses problemas, já que elas são capazes de extrair modelos e comportamentos da base de dados que dificilmente seriam percebidos por um observador humano. Esses modelos podem ser usados para identificar pardões em novas instâncias, automatizando o processo de monitoramento. O ganho que um profissional tem ao utilizar uma ferramenta dessa natureza é enorme: a tarefa de análise de bases de dados extensas é passada dele para o computador. O reconhecimento de padrões e capacidade de investigação de grandes bases de dados são os maiores desafios que profissionais da segurança da informação enfrentam. Nesse sentido, o objetivo é relevar à maquina cada vez mais o trabalho mecânico, para que o profissional possa se dedicar à análise crítica do material compilado.

Dentro da segurança da informação, escolheu-se um contexto mais específico para delimitar o escopo do trabalho: a detecção de fraude. Na legislação brasileira \cite[p. 324]{DePlacido1982}, a fraude é definida como ``o \emph{engano malicioso} ou a \emph{ação astuciosa}, promovidos de \emph{má fé}, para \emph{ocultação da verdade} ou \emph{fuga ao cumprimento do dever}''. Os tipos mais proeminentes de fraude ocorrem na área médica, nos seguros (de vida, casa, automóveis, etc), cartões de crédito e telecomunicações. Segundo a agência americana de investigação FBI \cite{FBI2010}, o custo das fraudes na área de seguros, excluindo-se os seguros de saúde, apenas nos Estados Unidos, é estimado em 40 bilhões de dólares por ano. O aumento médio nas apólices de uma família comum devido à fraude é calculado em 400 a 700 dólares.


Com frequência, nas ciências, pesquisadores se voltam a áreas diferentes das suas próprias, com o objetivo de encontrar soluções para os seus problemas adaptando outras soluções. Um exemplo disso, na Inteligência Artificial, são os Sistemas Imunológicos Artificiais, inspirados no sistema imunológico natural. Conforme definiu Castro:

\begin{quote}
``Sistemas Imunológicos Artificiais (SIA) são sistemas adaptativos, inspirados na teoria da imunidade e funções, princípios e modelos imunológicos observadas, que são aplicados à resolução de problemas.''\footnote{``Artificial Immune Systems (AIS) are adaptive systems, inspired by theoretical immunology and observed immune functions, principles and models, which are applied to problem solving.''. CASTRO, L. N.; TIMMIS, J. 2002. \emph{Artificial Immune Systems: A New Computational Intelligence Approach.} p. 57.}
\end{quote}

O sistema imunológico natural é um sistema robusto, complexo e adaptativo, que classifica as células do corpo como próprias e não-próprias. Isso é feito através de uma força de trabalho distribuída, capaz de agir de forma local ou global, usando uma rede de mensagens químicas para comunicar-se \cite{Aickelin2005}. Modelos computacionais foram criados para tentar adaptar o funcionamento do sistema imunológico natural a problemas da computação, e dá-se a esses modelos o nome de Sistemas Imunológicos Artificiais. A utilização desse tipo de sistema é relativamente nova: os primeiros trabalhos nessa área foram desenvolvidos a partir da metade dos anos 90. Mesmo assim é uma técnica que apresenta bons resultados e tem despertado o interesse de desenvolvedores de sistemas de segurança da informação. A modelagem desse tipo de sistema como um Sistema Imunológico Artificial oferece diversas vantagens:

\begin{itemize}
\item O sistema imunológico natural protege o corpo de invasões externas. Essa metáfora pode ser facilmente estendida para problemas de segurança, facilitando a modelagem dos componentes do sistema.
\item A natureza distribuída do sistema imunológico facilita a implementação do sistema seguindo o paradigma dos sistemas distribuídos da computação.
\item O sistema imunológico tem características que são muito importantes nos sistemas de segurança da informação. Ele contém uma memória, que reconhece agentes que o corpo já teve contato e apresenta um comportamento aptativo, capaz de reconhecer e combater novas ameaças que ainda não haviam sido tratadas.
\end{itemize}

Uma adição ainda mais recente foi a da Teoria do Perigo, proposta por Matzinger, em 1994. A principal revolução na Teoria do Perigo é a mudança na forma como as células tomam conhecimento e reagem, introduzindo a noção de perigo e tolerância. Essa mudança não é tanto na modelagem e representação dos dados, mas sim na decisão do que deve ser modelado, e o que deve gerar respostas do sistema \cite{Aickelin2005}. A adição da Teoria do Perigo a um sistema, além das vantagens proporcionadas pela modelagem como sistema imunológico artificial, restringe o domínio do problema, identificando o sub-conjunto de atributos mais importantes à detecção, e facilita a implementação em ambientes onde a definição de perigo muda com facilidade, características importantes em sistemas de seguraça da informação.

\section{Objetivos}
\section{Organização}

