\chapter{Planejamento do experimento}
\label{chap:prop}

Conforme apresentado nos capítulos anteriores, a modelagem de um sistema de detecção de fraude com a utilização de técnicas inspiradas no sistema imunológico natural busca trazer diversas vantagens tanto no processo de planejamento quanto na sua arquitetura final. Esse trabalho busca identificar exatamente \emph{quanto} esses modelos podem aperfeiçoar os métodos existentes. O método utilizado é a comparação dos resultados da classificação de algoritmos tradicionais e Sistemas Imunológicos Artificiais sobre um mesmo conjunto de dados. Esta seção apresenta os elementos principais para o desenvolvimento da proposta.

\iffalse
6.1 Etapas do experimento

- remove the word "desempenho" (find something better)
- remove "comparação de algoritmos", what's being compared are the results
- add step 0: select algorithms
- add step 2c: data preparation
- change "1. Calcular, [...]" to step 3
- put these three in a bigger step, called "execution"
- add a big step called "analysis"
    - 1. "tabular os resultados das execuções"
    - 2. "ordernar os resultados analisando os critérios"
    - 3. apresentar os resultados
- besides, go through chapter 7 and add any(/all =) step that was left out
\fi

\section{Comparação dos algoritmos}

A comparação do desempenho de diferentes algoritmos deve ser baseada em critérios sólidos para que tenha algum valor prático. Assim, nesse trabalho, a comparação dos algoritmos foi divida em duas etapas: execução e análise. A etapa de execução consistiu em:

\begin{enumerate}
    \item Para cada algoritmo:
        \begin{enumerate}
            \item Utilizar \emph{grid search} para otimizar os parâmetros para os dados.
            \item Utilizando esses parâmetros, executar os testes usando \emph{cross-validation}.
        \end{enumerate}
\end{enumerate}

Durante a etapa de execução, serão coletados dados sobre os resultados dos testes. Esses dados serão utilizados na etapa de análise, gerando, por fim, a comparação. A etapa de análise consistiu em:

\begin{enumerate}
    \item Calcular, para o resultado de cada algoritmo:
        \begin{enumerate}
            \item O erro médio relativo.
            \item O índice de Youden.
            \item O índice de Youden com significância maior para os falsos negativos.
        \end{enumerate}
\end{enumerate}

Conforme o padrão nesse tipo de experimento, os dados serão apresentados em formato de tabela e gráfico. Os critérios escolhidos já são parte integrante dos resultados exibidos na execução do WEKA. Essa escolha reflete tanto a aderência dessa plataforma aos padrões da área como a facilidade que uma ferramenta desse tipo oferece na determinação do método de avaliação.

\section{Descrição dos dados de teste}

Para os testes foram utilizados dois conjuntos de dados contendo informações de contas de cartões de crédito. Esses dados estão disponíveis publicamente e fazem parte de um projeto chamado StatLog \cite{Michie1994}. Esse projeto foi concebido para testar diversos métodos de classificação em problemas grandes e comercialmente relevantes, comparando os seus resultados e determinando o quanto eles atendiam as necessidades da indústria. Conforme a sua própria descrição, os objetivos do projeto eram três:

\begin{enumerate}[a)]
    \item Possibilitar medidas críticas de desempenho para procedimentos de classificação disponíveis,
    \item Indicar a natureza e escopo dos desenvolvimentos futuros necessários para que os métodos atendam as necessidades e expectativas dos usuários e
    \item Indicar as direções mais promissoras de desenvolvimento para abordagens comercialmente imaturas.
\end{enumerate}

Os conjuntos de dados são denominados German Credit (Cr.Ger) e Australian Credit (Cr.Aust). Ambos contém informações sobre contas de crédito, conforme apresentado nas próximas seções.

\subsection{Cr.Ger}

Esse conjunto de dados foi cedido pelo professor Hans Hoffman, da Universidade de Hamburgo. Ele contém 1000 instâncias. Os atributos e valores que esses atributos podem assumir são descritos na listagem \ref{lst:ge_dataset} (traduzido do original)\footnote{Na descrição dos atributos, DM significa \emph{deutsche mark} (marco alemão), a moeda corrente na Alemanha na época da coleta dos dados. Para efeito de comparação, o Banco Central Europeu estipulou a conversão irrevogável do marco alemão, a partir de 1º de janeiro de 1999, em DM 1.95583 = \euro 1 (http://www.ecb.int/press/pr/date/1998/html/pr981231$\textunderscore$2.en.html).}.

\vspace{0.5cm}
\begin{lstlisting}[caption=Atributos do conjunto de dados alemão,label=lst:ge_dataset]
    Atributo 1: (qualitativo)
    Situação da conta corrente existente
    A11 : ... < 0 DM
    A12 : 0 <= ... < 200 DM
    A13 : ... >= 200 DM
    A14 : sem conta corrente

    Atributo 2: (numérico)
    Duração em meses

    Atributo 3: (qualitativo)
    Histórico de crédito
    A30 : nenhum crédito retirado / todos os créditos pagos apropriadamente
    A31 : todos os créditos nesse banco pagos apropriadamente
    A32 : créditos existentes pagos apropriadamente até agora
    A33 : atraso no pagamento no passado
    A34 : conta crítica / outros créditos existentes (não nesse banco)

    Atributo 4: (qualitativo)
    Propósito
    A40 : carro (novo)
    A41 : carro (usado)
    A42 : móveis/equipamento
    A43 : rádio/televisão
    A44 : eletrodomésico
    A45 : reparos
    A46 : educação
    A47 : (férias - não existe no conjunto de dados)
    A48 : reciclagem profissional
    A49 : negócios
    A410 : outros

    Atributo 5: (numérico)
    Quantidade de crédito

    Atributo 6: (qualitativo)
    Poupança
    A61 : ... < 100 DM
    A62 : 100 <= ... < 500 DM
    A63 : 500 <= ... < 1000 DM
    A64 : .. >= 1000 DM
    A65 : desconhecido / sem poupança

    Atributo 7: (qualitativo)
    Emprego atual desde
    A71 : desempregado
    A72 : ... < 1 ano
    A73 : 1 <= ... < 4 anos
    A74 : 4 <= ... < 7 anos
    A75 : .. >= 7 anos

    Atributo 8: (numérico)
    Taxa de parcelamento em porcentagem do rendimento disponível

    Atributo 9: (qualitativo)
    Estado civil e sexo
    A91 : masculino : divorciado/separado
    A92 : feminino : divorciada/separada/casada
    A93 : masculino : solteiro
    A94 : masculino : casado/viúvo
    A95 : feminino : solteira

    Atributo 10: (qualitativo)
    Outros devedores / fiadores
    A101 : nenhum
    A102 : devedor solidário
    A103 : fiador

    Atributo 11: (numérico)
    Residência atual desde

    Atributo 12: (qualitativo)
    Propriedade
    A121 : imóvel
    A122 : se não A121 : financiamento / seguro de vida
    A123 : se não A121/A122 : carro ou outro, não incluso no atributo 6
    A124 : desconhecido / sem propriedade

    Atributo 13: (numérico)
    Idade em anos

    Atributo 14: (qualitativo)
    Outros planos de parcelamento
    A141 : banco
    A142 : lojas
    A143 : nenhum

    Atributo 15: (qualitativo)
    Residência
    A151 : alugada
    A152 : própria
    A153 : gratuita

    Atributo 16: (numérico)
    Número de créditos existentes nesse banco

    Atributo 17: (qualitativo)
    Emprego
    A171 : desempregado / sem proficiência / não-doméstico
    A172 : sem proficiência / doméstico
    A173 : proficiente / funcionário público
    A174 : administrador / auto-empregado /
           empregado altamente qualificado / oficial

    Atributo 18: (numérico)
    Número de dependentes

    Atributo 19: (qualitativo)
    Telefone
    A191 : nenhum
    A192 : sim, registrado sob o nom do consumidor

    Atributo 20: (qualitativo)
    Trabalhador estrangeiro
    A201 : sim
    A202 : não
\end{lstlisting}
\vspace{0.5cm}

\subsection{Cr.Aust}

Os atributos do conjunto de dados australiano são descritos na listagem \ref{lst:prop_au_dataset} (traduzido do original). Ele contém 690 instâncias. Uma grande desvantagem, muito comum nesse tipo de conjunto de dados (seção \ref{sec:fraud_data}), é a de o nome dos campos ter sido alterado, perdendo o significado original. Para garantir a privacidade das informações contidas no conjunto de dados, provavelmente por questões competitivas, a empresa que cedeu os dados utilizou um processo de anonimização, garantindo que as informações confidenciais não possam ser recuperadas.

Os dados ainda podem ser utilizados para a validação de sistemas de detecção, mas não é possível saber de que forma o sistema chegou ao diagnóstico, tornando o resultado bem menos útil.

\vspace{0.5cm}
\begin{lstlisting}[caption=Atributos do conjunto de dados Cr.Aust, label=lst:prop_au_dataset]
A1: b, a
A2: contínuo
A3: contínuo
A4: u, y, l, t
A5: g, p, gg
A6: c, d, cc, i, j, k, m, r, q, w, x, e, aa, ff
A7: v, h, bb, j, n, z, dd, ff, o
A8: contínuo
A9: t, f
A10: t, f
A11: contínuo
A12: t, f
A13: g, p, s
A14: contínuo
A15: contínuo
A16: +,- (atributo de classe)
\end{lstlisting}
\vspace{0.5cm}

\section{Algoritmos}

Os algoritmos utilizados no experimento são parte de um pacote de algoritmos desenvolvido por Jason Brownlee \cite{Brownlee2011}, na versão mais atual (1.8, maio de 2011). Esse pacote foi especialmente desenvolvido para a plataforma de aprendizagem de máquina WEKA (descrita na seção \ref{sec:prop_weka}) e são disponibilizados através de uma licença aberta (GNU GPL). Nele, são implementadas diversas categorias de algoritmos, como Redes Neurais e Sistemas Imunológicos Artificiais. A figura \ref{fig:prop_wekaais} mostra esses algoritmos conforme apresentados na interface do WEKA. Desses algoritmos, serão utilizados:

\vspace{0.5cm}
\begin{figure}[h]
    \centering
    \caption{Algoritmos no WEKA}
    \label{fig:prop_wekaais}
    \vspace{0.5cm}
    \includegraphics[width=0.75\textwidth]{img/weka_ais.png}
\end{figure}
\vspace{0.5cm}

\begin{enumerate}[a)]
    \item \textbf{Artificial Immune Recognition System (AIRS)}: algoritmo imunológico supervisionado descrito na seção \ref{sec:prop_airs}.
    \item \textbf{Clonal Selection Algorithm (CLONALG)}: um dos principais algoritmos imunológicos, o algoritmo da seleção clonal, descrito na seção \ref{sec:ais_clonalg}.
    \item \textbf{Immunos-81}: algoritmo imunológico descrito na seção \ref{sec:prop_immunos}.
    \iffalse adicionar os outros algoritmos do weka que serão utilizados (4.6) \fi
\end{enumerate}

\iffalse justificar \fi Implementações de todos esses algoritmos estão disponíveis na versão oficial da ferramenta WEKA. Isso facilita a comparação dos resultados com os resultados dos algoritmos imunológicos.

\section{WEKA}
\label{sec:prop_weka}

A plataforma de testes utilizado é o \emph{software} WEKA\nomenclature{WEKA}{Waikato Environment for Knowledge Analysis} (\emph{Waikato Environment for Knowledge Analysis}, Ambiente para Aprendizagem de Máquina de Waikato), uma suite de aplicações de aprendizagem de máquina desenvolvida na Universidade de Waikato na Nova Zelândia. Essa ferramenta é largamente utilizada em projetos nessa área, devido a sua licença aberta (GNU GPL\nomenclature{GNU GPL}{GNU General Public License}), que permite que seja utilizada quase sem restrições. A figura \ref{fig:prop_weka} mostra uma captura de tela do programa em execução, mostrando a visualização de um conjunto de dados de exemplo.

\vspace{0.5cm}
\begin{figure}[h]
    \centering
    \caption{Janela do módulo Explorer - WEKA 3.6.4}
    \label{fig:prop_weka}
    \vspace{0.5cm}
    \includegraphics[width=0.75\textwidth]{img/weka.png}
\end{figure}
\vspace{0.5cm}

\subsection{Arquivos ARFF}
\label{sec:prop_arff}

Para que um conjunto de dados possa ser utilizado no WEKA, ele deve estar em um formato específico, denominado ARFF (\emph{Attribute Relationship File Format}, Formato de Arquivo de Relação de Atributos). Um arquivo nesse formato é apresentado na listagem \ref{lst:prop_arff}. Esse é o mesmo arquivo apresentado na figura \ref{fig:prop_weka}, onde é mostrada a visualização dos dados pela interface gráfica.

\vspace{0.5cm}
\begin{lstlisting}[caption=Exemplo de arquivo no formato ARFF, label=lst:prop_arff]
    @relation weather

    @attribute outlook {sunny, overcast, rainy}
    @attribute temperature real
    @attribute humidity real
    @attribute windy {TRUE, FALSE}
    @attribute play {yes, no}

    @data
    sunny,85,85,FALSE,no
    sunny,80,90,TRUE,no
    overcast,83,86,FALSE,yes
    rainy,70,96,FALSE,yes
    rainy,68,80,FALSE,yes
    rainy,65,70,TRUE,no
    overcast,64,65,TRUE,yes
    sunny,72,95,FALSE,no
    sunny,69,70,FALSE,yes
    rainy,75,80,FALSE,yes
    sunny,75,70,TRUE,yes
    overcast,72,90,TRUE,yes
    overcast,81,75,FALSE,yes
    rainy,71,91,TRUE,no
\end{lstlisting}
\vspace{0.25cm}
\centerline{Fonte: \cite{Hall2009}}
\vspace{0.5cm}

Um arquivo ARFF é um arquivo de texto simples, em um formato padronizado, que combina a descrição dos dados e os dados em si \cite{Hall2009}. Ele é dividido em duas seções. A primeira, chamada de cabeçalho (\emph{header}) contém o nome do conjunto de dados (\emph{@relation}) e a lista de atributos (colunas) e seus tipos (\emph{@attribute}). Esses arquivos também podem conter comentários, que são ignorados quando o arquivo é analisado, e podem ser utilizados para adicionar informações ao arquivo como fonte, licença, etc. Um comentário é inserido através do caractere '\%', e tem efeito até o caractere de nova linha. Após o cabeçalho, vem a seção que contém os dados em si (\emph{@data}).

O nome do conjunto de dados é uma \emph{string} arbitrária, e serve apenas para a identificação na interface (figura \ref{fig:prop_weka}). A lista de atributos é uma sequência de entradas \emph{@attribute}. Cada atributo no conjunto de dados tem uma entrada \emph{@attribute} correspondente, que identifica o atributo. A ordem da declaração deve ser exatamente a ordem em que os atributos são listados na seção de dados. Isso facilita a listagem de dados, onde não é necessário indicar, para cada valor, o atributo ao qual ele se refere. O formato de cada \emph{@attribute} é:

\vspace{0.5cm}
\begin{lstlisting}[caption=Formato de uma entrada \emph{@attribute}, label=lst:prop_attribute_out]
@attribute nome tipo
\end{lstlisting}
\vspace{0.25cm}
\centerline{Fonte: \cite{Hall2009}}
\vspace{0.5cm}

O WEKA suporta quatro tipos de atributos. O formato ARFF, no entanto, suporta seis, e a conversão é feita automaticamente:

\begin{enumerate}[a)]
    \item \emph{numeric}: um atributo que pode ser tanto um número inteiro como real.
    \item \emph{inteiro}: tratado pelo WEKA como \emph{numeric}.
    \item \emph{real}: tratado pelo WEKA como \emph{numeric}.
    \item \emph{nominal}: um atributo com valores pré-definidos. Esses valores são especificados através de uma lista no formato: \{ valor1, valor2, valor3, \emph{...} \}.
    \item \emph{string}: uma \emph{string} de caracteres.
    \item \emph{date}: uma data, com um campo opcional que especifica o formato da data, como "yyyy-MM-dd HH:mm:ss".
\end{enumerate}

A seção de dados inicia com a entrada \emph{@data}, e segue até o final do arquivo. Nela são listados os valores do conjunto de dados. Cada entrada é representada em uma linha, ou seja, as entradas são separadas pelo caractere de nova linha. Em cada entrada, seus atributos são listados em uma lista separada por vírgulas. Como dito anteriormente, os atributos são identificados pela ordem ordem de declaração, logo ela deve ser a mesma no cabeçalho e nessa seção. Valores ausentes são representados por um caractere de ponto de interrogação (\emph{?}). Os atributos do tipo nominal e \emph{string} são \emph{case sensitive}, e devem ser envolvidos em aspas se contém espaços.

Por padrão, o último atributo é considerado o objetivo ou classe da instância (o atributo que é considerado uma função dos outros atributos). Isso pode ser alterado através da interface de configuração do experimento.

\subsection{Resultados}
\label{sec:prop_results}

A listagem \ref{lst:prop_weka_out} mostra um exemplo dos dados de saída após a execução de um teste de um dos algoritmos (LVQ) em um conjunto de dados de teste. Essa saída é dividia em seções. Na primeira seção, ``\emph{Run information}'', o atributo ``\emph{Scheme}'' mostra o classificador e todos os parâmetros correspondentes utilizados no treinamento. Os atributos ``\emph{Relation}'', ``\emph{Instances}'' e ``\emph{Atttributes}'' mostram informações do conjunto de dados: o nome, número de instâncias e atributos. Por fim, ``\emph{Test mode}'' mostra o método utilizado para a execução dos testes. As quatro opções disponíveis são:

\begin{enumerate}
    \item \emph{Use training set}: Executa os testes utilizando os mesmos dados utilizados no treinamento.
    \item \emph{Supplied test set}: Executa os testes utilizando um arquivo de testes fornecido.
    \item \emph{Cross-validation}: Executa os testes utilizando \emph{\emph{k}-fold cross-validation} (seção \ref{sec:eval_cross_validation}). O número de \emph{folds} pode ser definido.
    \item \emph{Percentage split}: Executa os testes utilizando uma parte dos dados para testes e outra para o treinamento. A procentagem pode ser definida.
\end{enumerate}

A seção seguinte da saída, ``\emph{Classifier model}'', mostra dados sobre o processo de treinamento: tempos de execução e informações sobre o modelo gerado pelo classificador. A seção ``\emph{Summary}'' mostra os resultados dos testes. Os dados dessa seção são: número de instâncias corretamente classificadas, número de instâncias incorretamente classificadas, o coeficiente \emph{kappa}, o erro de aproximação e a raiz quadrada do erro de aproximação, o erro relativo e a raiz quadrada do erro relativo e o número total de instâncias.

A seção ``\emph{Detailed class by accuracy}'' agrupa os resultados de cada classe presente nos dados: taxa de verdadeiros e falsos positivos, \emph{precision}, \emph{recall} e \emph{f-measure} e a área sob a curva ROC.

A última seção, ``\emph{Confusion matrix}'', é uma matriz de confusão. As colunas dessa matriz mostram a classificação das instâncias e as linhas, a classe verdadeira. Dessa forma, é possível identificar o número de instâncias correta e incorretamente classificadas. Para um conjunto de dados que contém apenas duas classes, essa tabela também mostra o número de falsos positivos e negativos.

\vspace{0.5cm}
\begin{lstlisting}[caption=Exemplo de saída de uma execução do WEKA, label=lst:prop_weka_out]
=== Run information ===

Scheme:       weka.classifiers.neural.lvq.Lvq1 -M 1 -C 20 -I 1000 -L 1 -R 0.3 \
    -S 1 -G false
Relation:     weather
Instances:    14
Attributes:   5
              outlook
              temperature
              humidity
              windy
              play
Test mode:    10-fold cross-validation

=== Classifier model (full training set) ===

-- Training Time Breakdown --
Model Initialisation Time   : 4ms
Model Training Time         : 43ms
Total Model Preparation Time: 47ms

-- Cass Distribution --
yes :  15 (75%)
no :  5 (25%)



Time taken to build model: 0.05 seconds

=== Stratified cross-validation ===
=== Summary ===

Correctly Classified Instances           5               35.7143 %
Incorrectly Classified Instances         9               64.2857 %
Kappa statistic                         -0.4651
Mean absolute error                      0.6429
Root mean squared error                  0.8018
Relative absolute error                135      %
Root relative squared error            162.5137 %
Total Number of Instances               14

=== Detailed Accuracy By Class ===

               TP Rate   FP Rate   Precision   Recall  F-Measure   ROC Area  Class
                 0.556     1          0.5       0.556     0.526      0.278    yes
                 0         0.444      0         0         0          0.278    no
Weighted Avg.    0.357     0.802      0.321     0.357     0.338      0.278

=== Confusion Matrix ===

 a b   <-- classified as
 5 4 | a = yes
 5 0 | b = no
\end{lstlisting}
\vspace{0.25cm}
\centerline{Fonte: \cite{Hall2009}}
\vspace{0.5cm}

\iffalse

\section{Método de pesquisa}

As atividades da segunda etapa do Trabalho de Conclusão de Curso serão desenvolvidas conforme apresentadas na tabela \ref{tab:prop_cron} e serão descritas à seguir.

A etapa de coleta e preparação de dados envolverá a adaptação dos dados nos conjuntos ao formato de entrada esperado pelo WEKA, já que eles se encontram em estado bruto. O formato mais comum para utilização nesse programa são arquivos \emph{Attribute Relationship File Format} (ARFF, Formato de Arquivo de Atributo-Relação). Um arquivo nesse formato é apresentado na listagem \ref{lst:prop_arff}. Esse mesmo arquivo foi apresentado na figura \ref{fig:prop_weka}, mostrando a visualização pela interface do WEKA.

\vspace{0.5cm}
\begin{lstlisting}[caption=Exemplo de arquivo no formato ARFF, label=lst:prop_arff]
    @relation weather

    @attribute outlook {sunny, overcast, rainy}
    @attribute temperature real
    @attribute humidity real
    @attribute windy {TRUE, FALSE}
    @attribute play {yes, no}

    @data
    sunny,85,85,FALSE,no
    sunny,80,90,TRUE,no
    overcast,83,86,FALSE,yes
    rainy,70,96,FALSE,yes
    rainy,68,80,FALSE,yes
    rainy,65,70,TRUE,no
    overcast,64,65,TRUE,yes
    sunny,72,95,FALSE,no
    sunny,69,70,FALSE,yes
    rainy,75,80,FALSE,yes
    sunny,75,70,TRUE,yes
    overcast,72,90,TRUE,yes
    overcast,81,75,FALSE,yes
    rainy,71,91,TRUE,no
\end{lstlisting}
\vspace{0.25cm}
\centerline{Fonte: \cite{Hall2009}}
\vspace{0.5cm}

Ainda, podem ser necessários ajustes específicos nos dados para que possam servir como entrada para alguns dos algoritmos, caso estes não tenham suporte aos tipos de dados. Com os dados prontos, iniciará a fase de execução dos diferentes algoritmos sobre esses conjuntos de dados. A coleta dos resultados será feita através dos dados de saída, mostrados na listagem \ref{lst:prop_weka_out}. Com os dados prontos, iniciará a fase de execução dos diferentes algoritmos sobre esses conjuntos de dados. A coleta dos resultados será feita através dos dados de saída, mostrados na listagem \ref{lst:prop_weka_out}.

Uma característica positiva da ferramenta WEKA é a possibilidade de alterar parâmetros tanto do algoritmo em si (caso eles ofereçam essa funcionalidade) quanto da execução do teste em si. Assim, pode-se executar diversos testes, variando esses parâmetros, para ao final comparar também essas configurações. A princípio, será definida apenas uma configuração, que será usada para a execução de todos os testes. No entanto, conforme for possível, poderão ser executados testes variando essa configuração, gerando comparação mais diversificadas.

De posse dos resultados dos testes de todos os algoritmos, iniciará a etapa de análise. Os resultados dos algoritmos imunológicos serão comparados com os resultados dos algoritmos tradicionais. Essa comparação envolve diversos níveis: taxas de erros e acertos (conforme apresentado na seção \ref{sec:eval_fraud}), tempos de execução (que inclui tempo de treinamento do algoritmo e o tempo dos testes), utilização de recursos, etc. Para a visualização dos resultados, serão criados elementos visuais, como tabelas e gráficos.

Esse período também inclui a documentação do processo através da redação da monografia e a apresentação ao fim do semestre.

\vspace{0.5cm}
\begin{table}[h]
    \centering
    \caption{Cronograma}
    \label{tab:prop_cron}
    \vspace{0.5cm}
    \begin{tabular}{l >{\arraybackslash}m{5cm} >{\centering\arraybackslash}m{6cm} c}
        \multicolumn{2}{c}{Etapa} & Resultado & Data \\
        \hline
        1 & Coleta e preparação                 & Arquivos no formato esperado pelo WEKA & março/2013 \\
        2 & Aplicação dos algoritmos            & Resultados da execução (taxas de acerto, métricas, tempo de execução, listagem \ref{lst:prop_weka_out} & abril/2013 \\
        3 & Comparação e análise                & Ranking de resultados & maio/2013 \\
        4 & Conclusão e contribuições           & Análise & junho/2013 \\
        5 & Redação e revisão                   & & março-julho/2013 \\
        6 & Apresentação                        & & julho/2013 \\
    \end{tabular}
\end{table}
\vspace{0.5cm}

\fi
