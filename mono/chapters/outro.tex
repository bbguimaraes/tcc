\chapter{Conclusão}

O estudo dos algoritmos baseados no sistema imunológico continua sendo uma área de pesquisa extensa, mesmo existindo algoritmos determinísticos mais rápidos e com resultados melhores, ou seja, não há uma área onde esses algoritmos possam ser considerados como os métodos mais efetivos. Isso parece confirmar a afirmação inicial de que o sistema imunológico e os Sistemas Imunológicos Artificiais são um bons para a modelagem de sistemas computacionais. O estudo desses sistemas é interessante e pode ser mais simples do que o de métodos determinísticos tradicionais.

O fato de um método não ser considerado útil não é motivo para que cesse a pesquisa e o desenvolvimento nessa área. Garrett cita como exemplo o algoritmo de redes neurais, que em 1969 não eram efetivas, mas que, hoje, após pesquisas extensas, são um dos algoritmos imunológicos mais efetivos \cite{Garrett2005}.

As semelhanças entre os diversos modelos de Sistemas Imunológicos Artificiais sugerem que haverá, no futuro, duas grandes áreas de pesquisa: seleção negativa/teoria do perigo, buscando identificar situações de perigo; e redes imunológicas, buscando destruir essas ameaças.

Além disso, muitas características do sistema imunológico natural são ainda pouco exploradas pela computação, como o sistema imunológico inato (citado no início desse texto) e a utilização de ataques paralelos a invasões. Esses fatores combinados sugerem um futuro de possibilidades interessantes para a área.
